\documentclass[12pt]{article}
\usepackage{pmmeta}
\pmcanonicalname{NormalSection}
\pmcreated{2013-03-22 17:26:25}
\pmmodified{2013-03-22 17:26:25}
\pmowner{pahio}{2872}
\pmmodifier{pahio}{2872}
\pmtitle{normal section}
\pmrecord{10}{39820}
\pmprivacy{1}
\pmauthor{pahio}{2872}
\pmtype{Definition}
\pmcomment{trigger rebuild}
\pmclassification{msc}{53A05}
\pmclassification{msc}{26A24}
\pmclassification{msc}{26B05}
\pmrelated{SecondFundamentalForm}
\pmrelated{DihedralAngle}
\pmdefines{normal curvature}

% this is the default PlanetMath preamble.  as your knowledge
% of TeX increases, you will probably want to edit this, but
% it should be fine as is for beginners.

% almost certainly you want these
\usepackage{amssymb}
\usepackage{amsmath}
\usepackage{amsfonts}

% used for TeXing text within eps files
%\usepackage{psfrag}
% need this for including graphics (\includegraphics)
%\usepackage{graphicx}
% for neatly defining theorems and propositions
 \usepackage{amsthm}
% making logically defined graphics
%%%\usepackage{xypic}

% there are many more packages, add them here as you need them

% define commands here

\theoremstyle{definition}
\newtheorem*{thmplain}{Theorem}

\begin{document}
\PMlinkescapeword{curvature}

\textbf{Normal sections}

Let $P$ be a point of a surface
\begin{align}
F(x,\,y,\,z) = 0,
\end{align}
where $F$ has the continuous first and \PMlinkescapetext{second order} partial derivatives in a neighbourhood of $P$.\, If one intersects the surface with a plane containing the surface normal at $P$, the intersection curve is called a {\em normal section}.\\

\textbf{Normal curvatures}

When the direction of the intersecting plane is varied, one gets different normal sections, and their \PMlinkname{curvatures}{CurvaturePlaneCurve} at $P$, the so-called {\em normal curvatures}, vary having a minimum value $\varkappa_1$ and a maximum value $\varkappa_2$.\, The arithmetic mean of $\varkappa_1$ and $\varkappa_2$ is called the {\em mean curvature} of the surface at $P$.\\

By the suppositions on the function $F$, examining the normal curvatures can without loss of generality be \PMlinkescapetext{reduced} to the following:\, Examine the curvature of the normal sections through the origin, the surface given in the form
\begin{align}
                        z = z(x,\,y),
\end{align}
where\, $z(x,\,y)$\, has the continuous first and \PMlinkescapetext{second order} partial derivatives in a neighbourhood of the origin and 
  $$z(0,\,0) = z'_x(0,\,0) = z'_y(0,\,0) = 0.$$
Indeed, one can take a new rectangular coordinate system with $P$ the new origin and the normal at $P$ the new $z$-axis; then the new $xy$-plane coincides with the tangent plane of the surface (1) at $P$.  The equation (1) defines the function of (2).



%%%%%
%%%%%
\end{document}
