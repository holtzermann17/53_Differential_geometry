\documentclass[12pt]{article}
\usepackage{pmmeta}
\pmcanonicalname{CircularHelix}
\pmcreated{2013-03-22 13:23:25}
\pmmodified{2013-03-22 13:23:25}
\pmowner{rspuzio}{6075}
\pmmodifier{rspuzio}{6075}
\pmtitle{circular helix}
\pmrecord{13}{33928}
\pmprivacy{1}
\pmauthor{rspuzio}{6075}
\pmtype{Definition}
\pmcomment{trigger rebuild}
\pmclassification{msc}{53A04}
\pmrelated{SpaceCurve}
\pmrelated{RightHandedSystemOfVectors}
\pmdefines{circular helices}

\usepackage{amsmath}
\usepackage{amsfonts}
\usepackage{amssymb}
\usepackage{graphicx}
\newcommand{\reals}{\mathbb{R}}
\newcommand{\natnums}{\mathbb{N}}
\newcommand{\cnums}{\mathbb{C}}
\newcommand{\znums}{\mathbb{Z}}
\newcommand{\lp}{\left(}
\newcommand{\rp}{\right)}
\newcommand{\lb}{\left[}
\newcommand{\rb}{\right]}
\newcommand{\supth}{^{\text{th}}}
\newtheorem{proposition}{Proposition}
\newtheorem{definition}[proposition]{Definition}

\newtheorem{theorem}[proposition]{Theorem}

\newcommand{\be}{\mathbf{e}}
\newcommand{\bg}{\boldsymbol{\gamma}}
\newcommand{\dbg}{\bg'}
\newcommand{\ddbg}{\bg''}
\newcommand{\dddbg}{\bg'''}
\newcommand{\der}[1]{#1{}'}
\newcommand{\bT}{\mathbf{T}}
\newcommand{\bN}{\mathbf{N}}
\newcommand{\bB}{\mathbf{B}}
\begin{document}
The space curve traced out by the parameterization
$$\bg(t)=\left[\begin{array}{c}a \cos(t)\\ a\sin
    (t)\\ bt\end{array}\right],\quad t\in \reals,\; a,b\in\reals$$
is called a \emph{circular helix} (plur. {\em helices}).  

Its Frenet frame is:
\begin{align*}
  \bT &= \frac{1}{\sqrt{a^2+b^2}} \begin{bmatrix} -a\sin t \\
  \hphantom{-}a\cos t \\b\end{bmatrix}\,,\\
  \bN &= \begin{bmatrix} -\cos t \\ -\sin t \\ 0 \end{bmatrix}\,,\\
  \bB &= \frac{1}{\sqrt{a^2+b^2}} \begin{bmatrix} \hphantom{-}b\sin t \\
  -b\cos t \\ a\end{bmatrix}\,.
\end{align*}
Its curvature and torsion are the following constants:
\begin{align*}
  \kappa = \frac{a}{a^2+b^2}\,, \quad
  \tau = \frac{b}{a^2+b^2}\,.
\end{align*}

A circular helix can be conceived of as a space curve with constant,
non-zero curvature, and constant, non-zero torsion.  Indeed, one can
show that if a space curve satisfies the above constraints, then there
exists a system of Cartesian coordinates in which the curve has a
parameterization of the form shown above.

\begin{figure}
\begin{center}
\includegraphics{helix2.eps}
\caption{A plot of a circular helix with $a = b = 1$, and $\kappa = \tau = 1/2$.}
\end{center}
\end{figure}

An important property of the circular helix is that for any point of it, the angle $\varphi$ between its tangent and the helix axis is constant. Indeed, if we consider the position vector of that arbitrary point, we have (where $\mathbf{k}$ is the unit vector parallel to helix axis)
\begin{align*}
\frac{d\bg}{dt}\cdot\mathbf{k}=\begin{bmatrix} -a\sin t \\
\hphantom{-}a\cos t \\ b \end{bmatrix}\begin{bmatrix} 0\; 0\; 1\end{bmatrix}=b
\equiv \bigg\lVert\frac{d\bg}{dt}\bigg\rVert\cos\varphi=\sqrt{a^2+b^2}\cos\varphi.
\end{align*} 
Therefore,
\begin{align*}
\cos\varphi=\frac{b}{\sqrt{a^2+b^2}} \text{constant},
\end{align*}
as was to be shown. 

There is also another parameter, the so-called \emph{pitch of the helix} $P$ which is the separation between two consecutive turns. 
(It is mostly used in the manufacture of screws.)
Thus,
\begin{align*}
P=\gamma_3(t+2\pi)-\gamma_3(t)= b(t+2\pi)-bt=2\pi b\,,
\end{align*}
and $P$ is also a constant.


%%%%%
%%%%%
\end{document}
