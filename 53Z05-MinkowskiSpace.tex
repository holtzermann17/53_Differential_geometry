\documentclass[12pt]{article}
\usepackage{pmmeta}
\pmcanonicalname{MinkowskiSpace}
\pmcreated{2013-03-22 16:16:51}
\pmmodified{2013-03-22 16:16:51}
\pmowner{cvalente}{11260}
\pmmodifier{cvalente}{11260}
\pmtitle{Minkowski space}
\pmrecord{7}{38393}
\pmprivacy{1}
\pmauthor{cvalente}{11260}
\pmtype{Definition}
\pmcomment{trigger rebuild}
\pmclassification{msc}{53Z05}
\pmrelated{PseudoRiemannianManifold}

\endmetadata

% this is the default PlanetMath preamble.  as your knowledge
% of TeX increases, you will probably want to edit this, but
% it should be fine as is for beginners.

% almost certainly you want these
\usepackage{amssymb}
\usepackage{amsmath}
\usepackage{amsfonts}

% used for TeXing text within eps files
%\usepackage{psfrag}
% need this for including graphics (\includegraphics)
%\usepackage{graphicx}
% for neatly defining theorems and propositions
%\usepackage{amsthm}
% making logically defined graphics
%%%\usepackage{xypic}

% there are many more packages, add them here as you need them

% define commands here

\begin{document}
\emph{Minkowski space} is a 4 dimensional real vector space with a non-degenerate pseudo-metric of signature $(-+++)$.

More precisely, $M$ with a metric $g$ is a Minkowski space iff:

\begin{itemize}

\item $M$ is a 4 dimensional real vector space
\item $g$ is a symmetric 2-covariant tensor (defines a quadratic form )
\item $g$ is non-degenerate (i.e. $\forall {x\in M}, g(x,y) = 0 \implies y=0$)
\item the \PMlinkname{diagonalization}{SylvestersLaw} of $g$ contains one negative and three positive diagonal entries\footnote{this convention is sometimes reversed depending on notation}.

\end{itemize}
%%%%%
%%%%%
\end{document}
