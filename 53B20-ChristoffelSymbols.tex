\documentclass[12pt]{article}
\usepackage{pmmeta}
\pmcanonicalname{ChristoffelSymbols}
\pmcreated{2013-03-22 15:43:52}
\pmmodified{2013-03-22 15:43:52}
\pmowner{juanman}{12619}
\pmmodifier{juanman}{12619}
\pmtitle{Christoffel symbols}
\pmrecord{24}{37681}
\pmprivacy{1}
\pmauthor{juanman}{12619}
\pmtype{Definition}
\pmcomment{trigger rebuild}
\pmclassification{msc}{53B20}
\pmclassification{msc}{53-01}
\pmsynonym{connection coefficients}{ChristoffelSymbols}
\pmrelated{Connection}

% this is the default PlanetMath preamble.  as your knowledge
% of TeX increases, you will probably want to edit this, but
% it should be fine as is for beginners.

% almost certainly you want these
\usepackage{amssymb}
\usepackage{amsmath}
\usepackage{amsfonts}

% used for TeXing text within eps files
%\usepackage{psfrag}
% need this for including graphics (\includegraphics)
%\usepackage{graphicx}
% for neatly defining theorems and propositions
%\usepackage{amsthm}
% making logically defined graphics
%%%\usepackage{xypic}

% there are many more packages, add them here as you need them

% define commands here
\begin{document}
A vector field in ${\mathbb{R}}^n$ can be seen as a differentiable ($C^{\infty}$) map 
$V\colon{\Bbb{R}}^n\to {\mathbb{R}}^n$. 

Or as a section ${\mathbb{R}}^n\stackrel{V}\to T({\mathbb{R}}^n)$
where $T{\mathbb{R}}^n\equiv{\mathbb{R}}^n\times{\mathbb{R}}^n$ is the ${\mathbb{R}}^n$'s trivial tangent bundle obeying
$p\mapsto (p,V(p)\in T_p({\mathbb{R}}^n))$ with $T_p({\mathbb{R}}^n)\equiv{\mathbb{R}}^n$ being the tangent space at $p$. 

Another viewpoint about tangent vectors is that they are also linear operators called {\bf derivations} and they act 
over scalars $f\colon {\mathbb{R}}^n\to{\mathbb{R}}$ via $p\mapsto Vf|_p=V(p)\cdot\nabla f|_p$.

Let $X$ be one of them and $dX|_p$ its Jacobian matrix evaluated at the point $p\in{\Bbb{R}}^n$.
Then, for any other vector field $Y\colon{\mathbb{R}}^n\to{\mathbb{R}}^n$, 
$$dX|_p(Y(p))$$
 measures how $X$ varies in the direction $Y$ at $p$. 

We have $dX|_p(Y(p))=(Y(p)\cdot\nabla X^1|_p,...,Y(p)\cdot\nabla X^n|_p)$, where $X=\sum_sX^se_s$ in components. 
Also, it is obvious that $p\mapsto dX|_p(Y(p))$ defines a new vector field in ${\mathbb{R}}^n$ which is symbolized as
$$D_YX$$
We can be consider it as a bilinear map
$$D:T({\mathbb{R}}^n)\times T({\mathbb{R}}^n)\to T({\mathbb{R}}^n).$$
$$(X,Y)\mapsto D_XY$$
Further, it is easy to see that for any scalar $f\colon {\mathbb{R}}^n\to{\mathbb{R}}$ 
\begin{enumerate}
\item $D_{fY}X=fD_YX$
\item $D_Y(fX)=(Yf)X+ fD_YX$
\item $D_XY-D_YX=[X,Y]$
\item $X(Y\cdot Z)=D_XY\cdot Z+X\cdot D_XZ$

\end{enumerate} 
Here we have abbreviated (as usual) $Yf=Y\cdot\nabla F$ and the operation $[X,Y]$ is the Lie bracket.

This $D$ is called the {\bf standard connection} of ${\mathbb{R}}^n$.

Now, let $M$ be a n-dimensional differentiable manifold and let $TM$ be its tangent bundle.
The set of differentiable sections $\Gamma(M)=\{X\colon M\to TM\}$ is a differentiable Lie algebra which is endowed with a differentiable inner product $g\colon\Gamma(M)\times\Gamma(M)\to{\mathbb{R}}$ via
$$g(X,Y)|_p=X(p)\cdot Y(p)$$ 
in each $T_p(M)\equiv{\mathbb{R}}^n$.

It is possible construct a bilinear operator $\nabla$
$$\nabla\colon \Gamma(M)\times\Gamma(M)\to\Gamma(M)$$
compatible with $g$ and which satisfies the following properties

\begin{enumerate}
\item $\nabla_{fY}X=f\nabla_YX$
\item $\nabla_Y(fX)=(Yf)X+ f\nabla_YX$
\item $\nabla_XY-\nabla_YX=[X,Y]$
\item $Xg(Y,Z)=g(\nabla_XY,Z)+g(X,\nabla_XZ)$
\end{enumerate} 

The {\bf Fundamental Theorem of Riemannian Geometry} establishes that this $\nabla$ exists and it is unique, 
and it is called the {\bf Levi-Civita connection} for the metric $g$ on $M$.

Now, if one uses a coordinated patch in $M$ one has a set of n-coordinated vector fields $\partial_1,..,\partial_n$
meaning $\partial_i={{\partial}\over{\partial u^i}}$ being $u^i$ the coordinate functions.
These are also dubbed holonomic derivations.

So it makes sense to speak about the derivatives $\nabla_{\partial_i}\partial_j$
and since the $\partial_i$ are tangent which generate at a point $T_p(M)$, then $\nabla_{\partial_i}\partial_j$
is also tangent, so there are $n\times n$ numbers (functions if one varies position) $\Gamma^s_{ij}$ which enters 
in the relation
$$\nabla_{\partial_i}\partial_j=\sum_s\Gamma^s_{ij}\partial_s.$$
These coefficients $\Gamma^s_{ij}$ are called {\bf Christoffel symbols} and an easy calculation shows that
$$\Gamma^k_{ij}={1\over 2}\sum_sg^{ks}[g_{sj,i}+g_{is,j}-g_{ij,s}]$$ 
where  $g_{ij}=g(\partial_i,\partial_j)$, $g^{ij}$ are the entries of the matrix $[g_{ij}]^{-1}$ and 
$g_{ij,k}=\partial_k(g_{ij})$.

Routinely one can check that under a change of coordinates $u^i\to w^j$ these functions transform as
$$\bar{\Gamma}^i_{kl}=
{{\partial w^i}\over{\partial u^m}}{{\partial u^n}\over{\partial w^k}}{{\partial u^p}\over{\partial w^l}}
\Gamma^m_{np}+{{\partial}^2u^p\over{\partial w^k\partial w^l}}{{\partial w^i}\over{\partial u^p}}
$$
here we have used Einstein's sum convention ($m,n,p$-sums) and the term 
$${
{\partial}^2u^p\over{\partial w^k\partial w_l}
}
{
{\partial w^i}\over{\partial u^p}
}$$ 
shows that the $\Gamma^i_{kl}$ are not tensors.

For a proof please see the last part in:
\PMlinkexternal{http://planetmath.org/?op=getobj\&from=collab\&id=64}{http://planetmath.org/?op=getobj\&from=collab\&id=64}

\paragraph{Connection with base vectors.} 

Let us assume that coordinates $u^i$ are referred to a right-handed orthogonal Cartesian system with attached constant base vectors $\mathbf{e}_i\equiv\mathbf{e}^i$ and coordinates $w^j$ referred to a general curvilinear system attached to a local covariant base vectors $\mathbf{g}_j$ and local contravariant base vectors $\mathbf{g}^k$, both systems embedded in the Euclidean space $\mathbb{R}^n$. We shall also suppose diffeomorphic the transfomation $u^i\mapsto w^j$. Then, by definition 
\begin{align}
\mathbf{g}_j:=\frac{\partial u^i}{\partial w^j}\mathbf{e}_i\:, \qquad
\mathbf{g}^j:=\frac{\partial w^j}{\partial u^i}\mathbf{e}^i\:,
\end{align}
and its inverses
\begin{align}
\mathbf{e}_i=\mathbf{e}^i=\frac{\partial u^i}{\partial w^j}\mathbf{g}^j=
\frac{\partial w^j}{\partial u^i}\mathbf{g}_j\:.
\end{align}
Let us consider differentiation of base vectors $\mathbf{g}_j$, which may be written from (1),(2)
\begin{align*}
\frac{\partial\mathbf{g}_j}{\partial w^k}=
\frac{\partial^2 u^i}{\partial w^j\partial w^k}\mathbf{e}_i=
\frac{\partial^2 u^i}{\partial w^j\partial w^k}
\frac{\partial u^i}{\partial w^s}\mathbf{g}^s=
\frac{\partial^2 u^i}{\partial w^j\partial w^k}
\frac{\partial w^s}{\partial u^i}\mathbf{g}_s\equiv
\frac{\partial\mathbf{g}_k}{\partial w^j}\:,
\end{align*}
and using the Christoffel symbols this becomes
\begin{align}
\frac{\partial\mathbf{g}_j}{\partial w^k}=\Gamma_{jks}\mathbf{g}^s=
\Gamma^r_{jk}\mathbf{g}_r\:,
\end{align}
where
\begin{align}
\Gamma_{jks}=\frac{\partial^2 u^i}{\partial w^j\partial w^k}
\frac{\partial u^i}{\partial w^s}\:, \qquad \Gamma^r_{jk}=g^{rs}\Gamma_{jks}\:.
\end{align}
Since the transformation of covariant and contravariant metric tensors are given by
\begin{align*}
g_{jk}=\frac{\partial u^i}{\partial w^j}\frac{\partial u^l}{\partial w^k}\delta_{il}\:,
\qquad g^{jk}=\frac{\partial w^j}{\partial u^i}
\frac{\partial w^k}{\partial u^l}\delta^{il}\:,
\end{align*}
is easy to see from here that Christoffel symbol $\Gamma_{jks}$ enjoy the property
\begin{align}
\Gamma_{jks}=\frac{1}{2}\bigg(\frac{\partial g_{js}}{\partial w^k}+
\frac{\partial g_{ks}}{\partial w^j}-
\frac{\partial g_{jk}}{\partial w^s}\bigg)\:\cdot
\end{align}
In a similar way we find for the derivative of the contravariant base vectors
\begin{align}
\frac{\partial\mathbf{g}^j}{\partial w^k}=-\Gamma^j_{ks}\mathbf{g}^s\:.
\end{align}
Is easy to show the following results:
\begin{align*}
\Gamma_{jks}=\Gamma_{kjs}=
\mathbf{g}_s\cdot\frac{\partial\mathbf{g}_k}{\partial w^j}=
\mathbf{g}_s\cdot\frac{\partial\mathbf{g}_j}{\partial w^k}\:,
\end{align*}
\begin{align*}
\Gamma^r_{jk}=\Gamma^r_{kj}=\mathbf{g}^r\cdot
\frac{\partial\mathbf{g}_j}{\partial w^k}=
\mathbf{g}^r\cdot
\frac{\partial\mathbf{g}_k}{\partial w^j}=
-\mathbf{g}_j\cdot\frac{\partial\mathbf{g}^r}{\partial w^k}\:,
\end{align*}
\begin{align*}
\Gamma^i_{ir}=\frac{1}{2}g^{is}(g_{is,r}+g_{rs,i}-g_{ir,s})=
\frac{1}{2}g^{is}g_{is,r}=
\frac{1}{2g}\frac{\partial g}{\partial g_{is}}
\frac{\partial g_{is}}{\partial w^r}=
\frac{1}{\sqrt{g}}\frac{\partial\sqrt{g}}{\partial w^r}\:,
\end{align*}
\begin{align*}
\Gamma_{jsk}+\Gamma_{ksj}=g_{jk,s}\:,
\end{align*}
comma denoting differentiation with respect to the curvilinear coordinates $w^j$ and $g=|g_{jk}|$. When the coordinate curves are orthogonal we have the following formulae for the Christoffel symbols: \bf{(repeated indices are not to be summed)}
\begin{align*}
\Gamma_{jks}=0\:, \qquad \Gamma^s_{jk}=0\:, \qquad (j\neq k\neq s\neq j),
\end{align*}
\begin{align*}
\Gamma_{iir}=-\frac{1}{2}\frac{\partial g_{ii}}{\partial w^r}\:, \qquad
\Gamma^r_{ii}=-\frac{1}{2g_{rr}}\frac{\partial g_{ii}}{\partial w^r}\:, \qquad
(r\neq i)\:,
\end{align*}
\begin{align*}
\Gamma_{iri}=\Gamma_{rii}=\frac{1}{2}\frac{\partial g_{ii}}{\partial w^r}\:,
\qquad \Gamma^r_{ri}=\Gamma^r_{ir}=
\frac{1}{2g_{rr}}\frac{\partial g_{rr}}{\partial w^i}=
\frac{1}{2}\frac{\partial\log{g_{rr}}}{\partial w^i}\:\cdot
\end{align*}
%%%%%
%%%%%
\end{document}
