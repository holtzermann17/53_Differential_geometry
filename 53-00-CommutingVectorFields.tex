\documentclass[12pt]{article}
\usepackage{pmmeta}
\pmcanonicalname{CommutingVectorFields}
\pmcreated{2013-03-22 15:22:37}
\pmmodified{2013-03-22 15:22:37}
\pmowner{matte}{1858}
\pmmodifier{matte}{1858}
\pmtitle{commuting vector fields}
\pmrecord{6}{37205}
\pmprivacy{1}
\pmauthor{matte}{1858}
\pmtype{Definition}
\pmcomment{trigger rebuild}
\pmclassification{msc}{53-00}

\endmetadata

% this is the default PlanetMath preamble.  as your knowledge
% of TeX increases, you will probably want to edit this, but
% it should be fine as is for beginners.

% almost certainly you want these
\usepackage{amssymb}
\usepackage{amsmath}
\usepackage{amsfonts}
\usepackage{amsthm}

\usepackage{mathrsfs}

% used for TeXing text within eps files
%\usepackage{psfrag}
% need this for including graphics (\includegraphics)
%\usepackage{graphicx}
% for neatly defining theorems and propositions
%
% making logically defined graphics
%%%\usepackage{xypic}

% there are many more packages, add them here as you need them

% define commands here

\newcommand{\sR}[0]{\mathbb{R}}
\newcommand{\sC}[0]{\mathbb{C}}
\newcommand{\sN}[0]{\mathbb{N}}
\newcommand{\sZ}[0]{\mathbb{Z}}

 \usepackage{bbm}
 \newcommand{\Z}{\mathbbmss{Z}}
 \newcommand{\C}{\mathbbmss{C}}
 \newcommand{\F}{\mathbbmss{F}}
 \newcommand{\R}{\mathbbmss{R}}
 \newcommand{\Q}{\mathbbmss{Q}}



\newcommand*{\norm}[1]{\lVert #1 \rVert}
\newcommand*{\abs}[1]{| #1 |}



\newtheorem{thm}{Theorem}
\newtheorem{defn}{Definition}
\newtheorem{prop}{Proposition}
\newtheorem{lemma}{Lemma}
\newtheorem{cor}{Corollary}
\begin{document}
Vector fields $X$, $Y$ on a manifold are \emph{commuting}
at $p\in M$ if 
$$
  [X,Y]_p=0
$$
where $[\cdot,\cdot]$ is the Lie bracket.

If $S$ is a subset of $M$, then we say that vector fields $X$ and $Y$ commute on $S$ if they commute at every pont of $S$.  In the case where $S = M$, i.e. when the vector fields commute at every point of the manifold, then we simply say that $X$ and $Y$ are commute.

A set $V$ of vector fields on a manifold is said to be commuting on a set $S$ if, for every pair of vector fields $A \in V$ and $B \in V$, it is the case that $A$ and $B$ commute.

If $S$ is an open set and $V$ is a set of commuting vector fields on $S$, then the cardinality of $V$ is not greater than the dimension of the manifold and one can find a local coordinate system about any point of $S$ for which these vector fields are coordinate vector fields.
%%%%%
%%%%%
\end{document}
