\documentclass[12pt]{article}
\usepackage{pmmeta}
\pmcanonicalname{IntegralManifold}
\pmcreated{2013-03-22 14:52:00}
\pmmodified{2013-03-22 14:52:00}
\pmowner{jirka}{4157}
\pmmodifier{jirka}{4157}
\pmtitle{integral manifold}
\pmrecord{6}{36542}
\pmprivacy{1}
\pmauthor{jirka}{4157}
\pmtype{Definition}
\pmcomment{trigger rebuild}
\pmclassification{msc}{53B25}
\pmclassification{msc}{52-00}
\pmclassification{msc}{37C10}
\pmrelated{FrobeniussTheorem}
\pmdefines{completely integrable}
\pmdefines{completely integrable distribution}

\endmetadata

% this is the default PlanetMath preamble.  as your knowledge
% of TeX increases, you will probably want to edit this, but
% it should be fine as is for beginners.

% almost certainly you want these
\usepackage{amssymb}
\usepackage{amsmath}
\usepackage{amsfonts}

% used for TeXing text within eps files
%\usepackage{psfrag}
% need this for including graphics (\includegraphics)
%\usepackage{graphicx}
% for neatly defining theorems and propositions
\usepackage{amsthm}
% making logically defined graphics
%%%\usepackage{xypic}

% there are many more packages, add them here as you need them

% define commands here
\theoremstyle{theorem}
\newtheorem*{thm}{Theorem}
\newtheorem*{lemma}{Lemma}
\newtheorem*{conj}{Conjecture}
\newtheorem*{cor}{Corollary}
\newtheorem*{example}{Example}
\theoremstyle{definition}
\newtheorem*{defn}{Definition}
\begin{document}
In the following we will \PMlinkescapetext{mean} $C^\infty$ when we say smooth.

\begin{defn}
Let $M$ be a smooth manifold of dimension $m$ and let $\Delta$ be a
distribution of dimension $n$ on $M$.  Suppose that $N$ is a connected
submanifold of $M$ such that for every $x \in N$ we have that
$T_x(N)$ (the tangent space of $N$ at $x$) is contained in $\Delta_x$
(the distribution at $x$).  We can abbreviate this by saying that
$T(N) \subset \Delta$.  We then say that $N$ is an {\em integral manifold}
of $\Delta$.
\end{defn}

Do note that $N$ could be of lower dimension then $\Delta$
and is not required to be a regular submanifold of $M$.

\begin{defn}
We say that a distribution $\Delta$ of dimension $n$ on $M$
is {\em completely integrable} if for
each point $x \in M$ there exists an integral manifold $N$ of $\Delta$ 
passing through $x$ such that the dimension of $N$ is equal to the dimension
of $\Delta$.
\end{defn}

An example of an integral manifold is the integral curve
of a non-vanishing vector field and then of course the span of the
vector field is a completely integrable distribution.

\begin{thebibliography}{9}
\bibitem{boothby}
William M.\@ Boothby.
{\em \PMlinkescapetext{An Introduction to Differentiable Manifolds and
Riemannian Geometry}},
Academic Press, San Diego, California, 2003.
\end{thebibliography}
%%%%%
%%%%%
\end{document}
