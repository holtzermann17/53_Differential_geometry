\documentclass[12pt]{article}
\usepackage{pmmeta}
\pmcanonicalname{FormulasInRiemannianGeometry}
\pmcreated{2013-03-22 15:32:55}
\pmmodified{2013-03-22 15:32:55}
\pmowner{matte}{1858}
\pmmodifier{matte}{1858}
\pmtitle{formulas in Riemannian geometry}
\pmrecord{8}{37446}
\pmprivacy{1}
\pmauthor{matte}{1858}
\pmtype{Definition}
\pmcomment{trigger rebuild}
\pmclassification{msc}{53B21}
\pmclassification{msc}{53B20}

\endmetadata

% this is the default PlanetMath preamble.  as your knowledge
% of TeX increases, you will probably want to edit this, but
% it should be fine as is for beginners.

% almost certainly you want these
\usepackage{amssymb}
\usepackage{amsmath}
\usepackage{amsfonts}
\usepackage{amsthm}

\newcommand{\bX}{\boldsymbol{X}}
\newcommand{\bY}{\boldsymbol{Y}}
\begin{document}
The aim of this page is to  collect
frequently used formulas in Riemannian geometry.

\subsubsection*{Symbol conventions.}
\begin{itemize}
\item $g_{ij}$ : the components of the
  metric tensor;
\item  $\Gamma_{ijk}=\Gamma_{jik}$ : the Christoffel symbols; 
\item $X_i=X^j g_{ij}$, and $Y^i$  : rank 1
  tensors;
\item $T_{ij}=T_{i}{}^{k} g_{jk}$ : a rank 2 tensor;
\item indices $i,j,k,l$ and subscripted versions thereof:  components
taken with respect to a local coordinate frame;
\item $x^i$, $y^j$ : systems of local coordinates;
\item $\partial_i = \frac{\partial}{\partial x^i}$ : local coordinate frame;
\item boldfaced symbols: the actual geometric quantity, rather
  than components; e.g.  $\bX = X^i \partial_i.$
\end{itemize}



\subsubsection*{Formulas for the covariant derivative.}
\begin{align*}
  \partial_k g_{ij} &= \Gamma_{kij} + \Gamma_{kji},\\
  \partial_k g^{ij} &= -(g^{jb} \Gamma_{bk}{}^i + g^{ia} \Gamma_{ak}{}^j), \\
  \nabla_k g_{ij} &= 0,\\
  \Gamma_{ijk} &= \tfrac{1}{2} ( \partial_i g_{jk} + \partial_j
  g_{ik}-\partial_k g_{ij}),\\
  \nabla_i X^j &= \partial_i X^j + \Gamma_{ik}{}^j X^k,\\
  \nabla_{\bX} \bY &= X^i \,\nabla_i Y^j\, \partial_j,\\
  \nabla_i X_j &= \partial_i X_j - \Gamma_{ij}{}^k X_k,\\
  \nabla_i T_{jk} &= \partial_i T_{jk} - \Gamma_{ij}{}^l T_{lk}
  -  \Gamma_{ik}{}^l T_{jl},\\
  \nabla_i T^j{}_k &= \partial_i T^j{}_k + \Gamma_{il}{}^j T^l{}_k -
  \Gamma_{ik}{}^l T^j{}_l.
\end{align*}

\subsubsection*{Formulas for geodesics}
A geodesic is a curve $c\colon I \to M$ satisfying
\[
\ddot{c}{}^i + \Gamma_{jk}{}^i\, \dot{c}^j\dot{c}^k = 0
\]
%%%%%
%%%%%
\end{document}
