\documentclass[12pt]{article}
\usepackage{pmmeta}
\pmcanonicalname{LocallyEuclidean}
\pmcreated{2013-03-22 14:14:49}
\pmmodified{2013-03-22 14:14:49}
\pmowner{matte}{1858}
\pmmodifier{matte}{1858}
\pmtitle{locally Euclidean}
\pmrecord{14}{35692}
\pmprivacy{1}
\pmauthor{matte}{1858}
\pmtype{Definition}
\pmcomment{trigger rebuild}
\pmclassification{msc}{53-00}
\pmrelated{Manifold}
\pmrelated{LocallyHomeomorphic}
\pmrelated{EmptyProduct}
\pmdefines{locally Euclidean space}
\pmdefines{chart}

\endmetadata

% this is the default PlanetMath preamble.  as your knowledge
% of TeX increases, you will probably want to edit this, but
% it should be fine as is for beginners.

% almost certainly you want these
\usepackage{amssymb}
\usepackage{amsmath}
\usepackage{amsfonts}

% used for TeXing text within eps files
%\usepackage{psfrag}
% need this for including graphics (\includegraphics)
%\usepackage{graphicx}
% for neatly defining theorems and propositions
%\usepackage{amsthm}
% making logically defined graphics
%%%\usepackage{xypic}

% there are many more packages, add them here as you need them

% define commands here

\newcommand{\sR}[0]{\mathbb{R}}
\newcommand{\sC}[0]{\mathbb{C}}
\newcommand{\sN}[0]{\mathbb{N}}
\newcommand{\sZ}[0]{\mathbb{Z}}

 \usepackage{bbm}
 \newcommand{\Z}{\mathbbmss{Z}}
 \newcommand{\C}{\mathbbmss{C}}
 \newcommand{\R}{\mathbbmss{R}}
 \newcommand{\Q}{\mathbbmss{Q}}



\newcommand*{\norm}[1]{\lVert #1 \rVert}
\newcommand*{\abs}[1]{| #1 |}
\begin{document}
A locally Euclidean space $X$ is a topological space that locally
``looks'' like $\R^n$.
This makes it possible to talk about 
coordinate axes around $X$. It also gives some topological structure
to the space: for example, since $\R^n$ is locally compact, so is $X$. 
However, the restriction does not induce any geometry onto $X$. 

\PMlinkescapeword{constant}
{\bf Definition} 
Suppose $X$ is a topological space. Then $X$ is
called \emph{locally Euclidean} if for each $x\in X$ there is a neighbourhood
 $U\subseteq X$, a $V\subseteq \sR^n$, and
a homeomorphism $\phi: U\to V$. Then the triple $(U,\phi, n)$
is called a \emph{chart} for $X$.

Here, $\sR$ is the set of real numbers, and for $n=0$ we define
$\sR^0$ as set with a single point equipped with the discrete topology.

\subsubsection*{Local dimension}
Suppose $X$ is a locally Euclidean space with $x\in X$. Further,
suppose $(U,\phi, n)$ is a chart of $X$ such that $x\in U$.
Then we define the \emph{local \PMlinkescapetext{dimension}} of $X$ at $x$ is $n$.
This is well defined, that is, the local dimension does not
depend on the chosen chart. If
$(U',\phi', n')$ is another chart with $x\in U'$, then
$\psi\circ \phi^{-1}: \phi(U\cap U') \to \psi(U\cap U')$
is a homeomorphism between $\phi(U\cap U')\subseteq \sR^n$
and $\psi(U\cap U')\subseteq \sR^{n'}$. By Brouwer's theorem
for the invariance of dimension (which is nontrivial),
it follows that $n=n'$.

If the local dimension is constant, say $n$, we say that the dimension
of $X$ is $n$, and write $\dim X = n$. 
\subsubsection*{Examples}
\begin{itemize}
\item Any set with the discrete topology, is a locally 
      Euclidean of dimension $0.$ 
\item Any open subset of $\sR^n$ is locally Euclidean.
\item Any manifold is locally Euclidean. For example, 
using a stereographic projection, one can show that the sphere $S^n$
is locally Euclidean.
\item The long line is locally Euclidean of dimension one.  Note that the long line is not Hausforff. \cite{conlon}.
\end{itemize}

\subsubsection*{Notes}
The concept locally Euclidean has a different meaning in the
setting of Riemannian manifolds. 

\begin{thebibliography}{9}
 \bibitem {conlon} L. Conlon, \emph{Differentiable Manifolds: A first course},
Birkh\"auser, 1993.
\end{thebibliography}
%%%%%
%%%%%
\end{document}
