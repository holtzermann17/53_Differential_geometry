\documentclass[12pt]{article}
\usepackage{pmmeta}
\pmcanonicalname{SemicubicalParabola}
\pmcreated{2013-03-22 18:29:18}
\pmmodified{2013-03-22 18:29:18}
\pmowner{pahio}{2872}
\pmmodifier{pahio}{2872}
\pmtitle{semicubical parabola}
\pmrecord{21}{41165}
\pmprivacy{1}
\pmauthor{pahio}{2872}
\pmtype{Definition}
\pmcomment{trigger rebuild}
\pmclassification{msc}{53A04}
\pmsynonym{semicubic parabola}{SemicubicalParabola}
\pmsynonym{Neile's parabola}{SemicubicalParabola}
\pmsynonym{Neil's parabola}{SemicubicalParabola}
\pmrelated{CubeOfANumber}
\pmrelated{OneSidedDerivatives}

\endmetadata

% this is the default PlanetMath preamble.  as your knowledge
% of TeX increases, you will probably want to edit this, but
% it should be fine as is for beginners.

% almost certainly you want these
\usepackage{amssymb}
\usepackage{amsmath}
\usepackage{amsfonts}

% used for TeXing text within eps files
%\usepackage{psfrag}
% need this for including graphics (\includegraphics)
%\usepackage{graphicx}
% for neatly defining theorems and propositions
 \usepackage{amsthm}
% making logically defined graphics
%%%\usepackage{xypic}
\usepackage{pstricks}
\usepackage{pst-plot}

% there are many more packages, add them here as you need them

% define commands here

\theoremstyle{definition}
\newtheorem*{thmplain}{Theorem}

\begin{document}
The curve \,$y = x^3$\, is sometimes called a {\em cubic(al) parabola}, as \,$y = x^2$\, a parabola.\, Thus there is reason to call the curve
$$y \;=\; x^{\frac{3}{2}} \;=\; x\sqrt{x}$$
a {\em semicubical parabola}.\, Since this equation implies that
\begin{align}
y^2 \;=\; x^3,
\end{align}
which is satisfied as well by negative real numbers $y$, it is natural \PMlinkescapetext{to join} to the semicubical parabola also points below the $x$-axis.\, Therefore the equation (1) can be equivalently written as
\begin{align}
y \;=\; \pm{x}^{\frac{3}{2}} \;=\; \pm{x}\sqrt{x}.
\end{align}
According to (1) or (2), the semicubical parabola has symmetry about the $x$-axis.\, Because the positive \PMlinkescapetext{branch} of (2) forms a strictly increasing and the negative branch a strictly decreasing power function for\, $x \leqq 0$,\, the graph of (2) has an ordinary cusp in the origin.


\begin{center}
\begin{pspicture}(-2,-6)(4,6)
\psaxes[Dx=1,Dy=1]{->}(0,0)(-1.5,-5.5)(3.8,5.5)
\rput(3.8,-0.2){$x$}
\rput(0.2,5.5){$y$}
\psplot[linecolor=blue]{0}{3}{x x sqrt mul}
\psplot[linecolor=blue]{0}{3}{0 x x sqrt mul sub}
\rput(3,2.5){$y^2 = x^3$}
\end{pspicture}
\end{center}

The arc length of the semicubical parabola from the origin to the abscissa $x$ is
$$s(x) \;=\; \int_0^x\sqrt{1+\!\left(\frac{d}{dt}t^\frac{3}{2}\right)^2}\,dt \;=\; 
\frac{(4\!+\!9x)\sqrt{4\!+\!9x}-8}{27},$$
i.e. an algebraic expression in $x$.\, Actually, the semicubical parabola is historically the first algebraic curve (after the straight line) having an algebraic arc length.\\

Making a \PMlinkname{squeezing}{SqueezingMathbbRn} of the plane, one can write a more general equation
\begin{align}
ay^2 \;=\; x^3
\end{align}
of the semicubical parabola; here $a$ is a positive constant.\\

The semicubic parabola is also the evolute of a parabola; e.g. the equation
$$3y^2 \;=\; (x-\frac{1}{2})^3$$
\PMlinkescapetext{represents} the evolute of the parabola\, $y^2 = x$\, (cf. determining envelope).


%%%%%
%%%%%
\end{document}
