\documentclass[12pt]{article}
\usepackage{pmmeta}
\pmcanonicalname{LieBracket}
\pmcreated{2013-03-22 14:10:02}
\pmmodified{2013-03-22 14:10:02}
\pmowner{rspuzio}{6075}
\pmmodifier{rspuzio}{6075}
\pmtitle{Lie bracket}
\pmrecord{10}{35591}
\pmprivacy{1}
\pmauthor{rspuzio}{6075}
\pmtype{Definition}
\pmcomment{trigger rebuild}
\pmclassification{msc}{53-00}
\pmrelated{HamiltonianAlgebroids}

\endmetadata

% this is the default PlanetMath preamble.  as your knowledge
% of TeX increases, you will probably want to edit this, but
% it should be fine as is for beginners.

% almost certainly you want these
\usepackage{amssymb}
\usepackage{amsmath}
\usepackage{amsfonts}

% used for TeXing text within eps files
%\usepackage{psfrag}
% need this for including graphics (\includegraphics)
%\usepackage{graphicx}
% for neatly defining theorems and propositions
%\usepackage{amsthm}
% making logically defined graphics
%%%\usepackage{xypic}

% there are many more packages, add them here as you need them

% define commands here

\newcommand{\sR}[0]{\mathbb{R}}
\newcommand{\sC}[0]{\mathbb{C}}
\newcommand{\sN}[0]{\mathbb{N}}
\newcommand{\sZ}[0]{\mathbb{Z}}

% The below lines should work as the command
% \renewcommand{\bibname}{References}
% without creating havoc when rendering an entry in 
% the page-image mode.
\makeatletter
\@ifundefined{bibname}{}{\renewcommand{\bibname}{References}}
\makeatother

\newcommand*{\norm}[1]{\lVert #1 \rVert}
\newcommand*{\abs}[1]{| #1 |}

\begin{document}
The Lie bracket is an anticommutative, bilinear, first order differential operator on vector fields.  It may be defined either in terms of local coordinates or in a global, coordinate-free fashion.  Though both defintions are prevalent, it is perhaps easier to formulate the Lie Bracket without the use of coordinates at all, as a commutator:

{\bf Definition} (Global, coordinate-free) Suppose $X$ and $Y$ are vector fields on a smooth manifold $M$.  Regarding these vector fields as operators on functions, the Lie bracket is their commutator:
\begin{align*}
[X,Y](f)=X(Y(f))-Y(X(f)).
\end {align*}

{\bf Definition} (Local coordinates) Suppose $X$ and $Y$ are vector fields on a smooth $n$-dimensional manifold $M$,
suppose $(x^1,\ldots, x^n)$ are local coordinates around some point $x\in M$, 
and suppose that in these local coordinates
\begin{eqnarray*}
X(x)&=&X^i(x) \frac{\partial}{\partial x^i}\Big|_x, \\
Y(x)&=&Y^i(x) \frac{\partial}{\partial x^i}\Big|_x.
\end{eqnarray*}
Then the \emph{Lie bracket} of the above vector fields is the locally defined vector field
$$[X,Y](x) = X^i \frac{\partial Y^j}{\partial x^i} \frac{\partial}{\partial x^j}\Big|_x-Y^i \frac{\partial X^j}{\partial x^i} \frac{\partial}{\partial x^j}\Big|_x.$$
(The Einstein summation convention employed in the above equations ---
repeated indices are to be summed from the range 1 to $n$.)

\subsubsection*{Properties}
Suppose $X,Y,Z$ are smooth vector fields on a smooth manifold $M$. 
\begin{itemize}
 \item $[X,Y]=\mathcal{L}_XY$ where $\mathcal{L}_XY$ is the Lie derivative.
 \item $[\cdot,\cdot]$ is anti-symmetric and bi-linear. 
\item Vector fields on $M$ with the Lie bracket is a Lie algebra. That is to say, the Lie bracket satisfies the Jacobi identity:
 \[ [X,[Y,Z]] + [Y,[Z,X]] + [Z,[X,Y]]=0. \]
\item The Lie bracket is covariant with respect to changes of coordinates.
 \end{itemize}

%%%%%
%%%%%
\end{document}
