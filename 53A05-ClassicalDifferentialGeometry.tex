\documentclass[12pt]{article}
\usepackage{pmmeta}
\pmcanonicalname{ClassicalDifferentialGeometry}
\pmcreated{2013-03-22 15:17:03}
\pmmodified{2013-03-22 15:17:03}
\pmowner{matte}{1858}
\pmmodifier{matte}{1858}
\pmtitle{classical differential geometry}
\pmrecord{39}{37074}
\pmprivacy{1}
\pmauthor{matte}{1858}
\pmtype{Topic}
\pmcomment{trigger rebuild}
\pmclassification{msc}{53A05}
\pmclassification{msc}{53A04}
\pmrelated{DifferentialGeometry}

% this is the default PlanetMath preamble.  as your knowledge
% of TeX increases, you will probably want to edit this, but
% it should be fine as is for beginners.

% almost certainly you want these
\usepackage{amssymb}
\usepackage{amsmath}
\usepackage{amsfonts}
\usepackage{amsthm}

\usepackage{mathrsfs}

% used for TeXing text within eps files
%\usepackage{psfrag}
% need this for including graphics (\includegraphics)
%\usepackage{graphicx}
% for neatly defining theorems and propositions
%
% making logically defined graphics
%%%\usepackage{xypic}

% there are many more packages, add them here as you need them

% define commands here

\newcommand{\sR}[0]{\mathbb{R}}
\newcommand{\sC}[0]{\mathbb{C}}
\newcommand{\sN}[0]{\mathbb{N}}
\newcommand{\sZ}[0]{\mathbb{Z}}

 \usepackage{bbm}
 \newcommand{\Z}{\mathbbmss{Z}}
 \newcommand{\C}{\mathbbmss{C}}
 \newcommand{\F}{\mathbbmss{F}}
 \newcommand{\R}{\mathbbmss{R}}
 \newcommand{\Q}{\mathbbmss{Q}}



\newcommand*{\norm}[1]{\lVert #1 \rVert}
\newcommand*{\abs}[1]{| #1 |}



\newtheorem{thm}{Theorem}
\newtheorem{defn}{Definition}
\newtheorem{prop}{Proposition}
\newtheorem{lemma}{Lemma}
\newtheorem{cor}{Corollary}
\begin{document}
\subsection*{Curves in $\R^2$}
\begin{itemize}
\item inflexion point
\item singular points of plane curve
\item isocline
\item curvature (plane curve)
\item circle of curvature
\item curvature determines the curve
\item curvature of Nielsen's spiral
\item osculating curve
\item orthogonal curves
\item isogonal trajectory
\item parallel curves
\item properties of parallel curves
\item evolute
\item evolute of cycloid
\item \PMlinkname{Serret-Frenet equations in  $\R^2$}{SerretFrenetEquationsInMathbbR2}
\item famous curves in the plane
\item arc-parametrizations 
\item envelope
\item determining envelope
\item catacaustic
\end{itemize}

\subsection*{Curves in $\R^3$}
\begin{itemize}
\item \PMlinkname{Serret-Frenet equations in  $\R^3$}{SerretFrenetFormulas}
\item space curve
\item level curve
\item \PMlinkname{curvature}{CurvatureOfACurve} and \PMlinkname{torsion}{Torsion} of a space curve
\item moving trihedron
\end{itemize}

\subsection*{Surfaces in $\R^3$}
\begin{itemize}
\item level curve, level surface
\item surface of revolution
\item surface normal
\item normal section
\item normal curvatures
\item Meusnier's theorem
\item mean curvature at surface point
\item \PMlinkname{first fundamental form}{FirstFundamentalForm}
\item second fundamental form
\item sphere map and shape operator
\item Gaussian curvature and mean curvature
\item geodesic
\item Gauss-Bonnet theorem
\item standard connection in $\R^3$
\item Gauss equation
\end{itemize}

\subsection*{The space $\R^3$}
\begin{itemize}
\item ortho-normal frame fields in $\R^3$ (or non constant ortho-normal triples of vector fields) 
\item rate of rotation of an o.f.f.
\item euclidean spin connection
\item $\R^3$ {\bf Cartan structural equations I, II}
\end{itemize}

\subsection*{Variational calculus}
\begin{itemize}
\item calculus of variations
\item classical isoperimetric problem
\item least surface of revolution
\item brachistochrone curve
\item equation of catenary via calculus of variations
\end{itemize}

%%%%%
%%%%%
\end{document}
