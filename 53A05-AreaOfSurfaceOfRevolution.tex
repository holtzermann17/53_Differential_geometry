\documentclass[12pt]{article}
\usepackage{pmmeta}
\pmcanonicalname{AreaOfSurfaceOfRevolution}
\pmcreated{2014-07-24 18:36:37}
\pmmodified{2014-07-24 18:36:37}
\pmowner{rspuzio}{6075}
\pmmodifier{pahio}{2872}
\pmtitle{area of surface of revolution}
\pmrecord{13}{37655}
\pmprivacy{1}
\pmauthor{rspuzio}{2872}
\pmtype{Topic}
\pmcomment{trigger rebuild}
\pmclassification{msc}{53A05}
\pmclassification{msc}{26B15}
\pmsynonym{area of revolution}{AreaOfSurfaceOfRevolution}
\pmsynonym{surface area of revolution}{AreaOfSurfaceOfRevolution}
\pmrelated{SurfaceOfRevolution2}
\pmrelated{VolumeOfSolidOfRevolution}

% this is the default PlanetMath preamble.  as your knowledge
% of TeX increases, you will probably want to edit this, but
% it should be fine as is for beginners.

% almost certainly you want these
\usepackage{amssymb}
\usepackage{amsmath}
\usepackage{amsfonts}

% used for TeXing text within eps files
%\usepackage{psfrag}
% need this for including graphics (\includegraphics)
%\usepackage{graphicx}
% for neatly defining theorems and propositions
%\usepackage{amsthm}
% making logically defined graphics
%%%\usepackage{xypic}

% there are many more packages, add them here as you need them

% define commands here
\begin{document}
A \emph{surface of revolution} is a 3D surface, generated when an arc is rotated fully around a straight line.

The general surface of revolution is obtained when the arc is rotated about an arbitrary axis. If one chooses Cartesian coordinates, and specializes to the case of a surface of revolution generated by rotating about the $x$-axis a curve described by $y$ in the interval $[a, b]$, its area can be calculated by the formula

$$A = 2 \pi \int_{a}^{b} y \, \sqrt{ 1 + \left(\frac{dy}{dx}\right)^2 } \, dx$$

Similarly, if the curve is rotated about the $y$-axis rather than the $x$-axis, one has the following formula:

$$A = 2 \pi \int_{a}^{b} x \, \sqrt{ 1 + \left(\frac{dx}{dy}\right)^2 } \, dy$$

The general formula is most often seen with parametric coordinates. If $x(t)$ and $y(t)$ describe the curve, and $x(t)$ is always positive or zero, then the area of the general surface of revolution $A$ in the interval $[a, b]$ can be calulated by the formula

$$A = 2 \pi \int_{a}^{b} y \, \sqrt{ \left(\frac{dx}{dt}\right)^2 + \left(\frac{dy}{dt}\right)^2 } \, dt$$

To obtain a specific surface of revolution, translation or rotation can be used to move an arc before revolving it around an axis. For example, the specific surface of revolution around the line $y = s$ can be found by replacing $y$ with $y\!-\!s$, moving the arc towards the $x$-axis so\, $y = s$\, lies on it. Now, the surface of revolution can be found using one of the formulae above.

In this specific case, replacing $y$ with\, $y = s$,\, the area of a surface of revolution is found using the formula

$$A = 2 \pi \int_{a}^{b} (y-s) \sqrt{ \left(\frac{dy}{dx}\right)^2 } \, dy$$
%%%%%
%%%%%
\end{document}
