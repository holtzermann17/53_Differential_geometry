\documentclass[12pt]{article}
\usepackage{pmmeta}
\pmcanonicalname{TopologicalManifold}
\pmcreated{2013-03-22 16:02:05}
\pmmodified{2013-03-22 16:02:05}
\pmowner{juanman}{12619}
\pmmodifier{juanman}{12619}
\pmtitle{topological manifold}
\pmrecord{7}{38081}
\pmprivacy{1}
\pmauthor{juanman}{12619}
\pmtype{Definition}
\pmcomment{trigger rebuild}
\pmclassification{msc}{53-00}
\pmclassification{msc}{57R50}
\pmrelated{Manifold}
\pmdefines{topological manifold}

% this is the default PlanetMath preamble.  as your knowledge
% of TeX increases, you will probably want to edit this, but
% it should be fine as is for beginners.

% almost certainly you want these
\usepackage{amssymb}
\usepackage{amsmath}
\usepackage{amsfonts}

% used for TeXing text within eps files
%\usepackage{psfrag}
% need this for including graphics (\includegraphics)
%\usepackage{graphicx}
% for neatly defining theorems and propositions
%\usepackage{amsthm}
% making logically defined graphics
%%%\usepackage{xypic}

% there are many more packages, add them here as you need them

% define commands here

\begin{document}
The definition given in the manifold article is in fact the definition of topological manifold. If we change it to allow only smooth (differentiable) transition (maximal rank) functions, then we speak of a smooth manifold.  
Or if we take real analytic function as the transition ones we get $\mathbb{R}$-analytic manifolds.

One must emphasize that the invariant number $n$ appearing in the definition there is the dimension of the manifold which is also the dimension of the tangent space at a point, in a smooth manifold.
%%%%%
%%%%%
\end{document}
