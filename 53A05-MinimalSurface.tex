\documentclass[12pt]{article}
\usepackage{pmmeta}
\pmcanonicalname{MinimalSurface}
\pmcreated{2013-03-22 18:08:56}
\pmmodified{2013-03-22 18:08:56}
\pmowner{pahio}{2872}
\pmmodifier{pahio}{2872}
\pmtitle{minimal surface}
\pmrecord{5}{40705}
\pmprivacy{1}
\pmauthor{pahio}{2872}
\pmtype{Definition}
\pmcomment{trigger rebuild}
\pmclassification{msc}{53A05}
\pmclassification{msc}{26B05}
\pmclassification{msc}{26A24}
\pmrelated{PlateausProblem}
\pmrelated{LeastSurfaceOfRevolution}

% this is the default PlanetMath preamble.  as your knowledge
% of TeX increases, you will probably want to edit this, but
% it should be fine as is for beginners.

% almost certainly you want these
\usepackage{amssymb}
\usepackage{amsmath}
\usepackage{amsfonts}

% used for TeXing text within eps files
%\usepackage{psfrag}
% need this for including graphics (\includegraphics)
%\usepackage{graphicx}
% for neatly defining theorems and propositions
 \usepackage{amsthm}
% making logically defined graphics
%%%\usepackage{xypic}

% there are many more packages, add them here as you need them

% define commands here

\theoremstyle{definition}
\newtheorem*{thmplain}{Theorem}

\begin{document}
Among the surfaces \,$F(x,\,y,\,z) = 0$,\, with $F$ twice continuously differentiable, a {\em minimal surface} is such that in every of its points, the mean curvature vanishes.\, Because the mean curvature is the arithmetic mean of the principal curvatures $\varkappa_1$ and $\varkappa_2$, the equation
$$\varkappa_2 \;=\; -\varkappa_1$$
is valid in each point of a minimal surface.

A minimal surface has also the property that every sufficiently little portion of it has smaller area than any other regular surface with the same boundary curve.

Trivially, a plane is a minimal surface.\, The catenoid is the only surface of revolution which is also a minimal surface.
%%%%%
%%%%%
\end{document}
