\documentclass[12pt]{article}
\usepackage{pmmeta}
\pmcanonicalname{IntegralCurve}
\pmcreated{2013-03-22 15:16:31}
\pmmodified{2013-03-22 15:16:31}
\pmowner{matte}{1858}
\pmmodifier{matte}{1858}
\pmtitle{integral curve}
\pmrecord{5}{37063}
\pmprivacy{1}
\pmauthor{matte}{1858}
\pmtype{Definition}
\pmcomment{trigger rebuild}
\pmclassification{msc}{53-00}

\endmetadata

% this is the default PlanetMath preamble.  as your knowledge
% of TeX increases, you will probably want to edit this, but
% it should be fine as is for beginners.

% almost certainly you want these
\usepackage{amssymb}
\usepackage{amsmath}
\usepackage{amsfonts}
\usepackage{amsthm}

\usepackage{mathrsfs}

% used for TeXing text within eps files
%\usepackage{psfrag}
% need this for including graphics (\includegraphics)
%\usepackage{graphicx}
% for neatly defining theorems and propositions
%
% making logically defined graphics
%%%\usepackage{xypic}

% there are many more packages, add them here as you need them

% define commands here

\newcommand{\sR}[0]{\mathbb{R}}
\newcommand{\sC}[0]{\mathbb{C}}
\newcommand{\sN}[0]{\mathbb{N}}
\newcommand{\sZ}[0]{\mathbb{Z}}

 \usepackage{bbm}
 \newcommand{\Z}{\mathbbmss{Z}}
 \newcommand{\C}{\mathbbmss{C}}
 \newcommand{\F}{\mathbbmss{F}}
 \newcommand{\R}{\mathbbmss{R}}
 \newcommand{\Q}{\mathbbmss{Q}}



\newcommand*{\norm}[1]{\lVert #1 \rVert}
\newcommand*{\abs}[1]{| #1 |}



\newtheorem{thm}{Theorem}
\newtheorem{defn}{Definition}
\newtheorem{prop}{Proposition}
\newtheorem{lemma}{Lemma}
\newtheorem{cor}{Corollary}
\begin{document}
{\bf Definition} 
Suppose $M$ is a smooth manifold, and $X$ is a 
smooth vector field on $M$. Then an {\bf integral curve} of $X$ through 
a point $x\in M$ is a curve $c\colon I\to M$, such that 
\begin{eqnarray*}
 c'(t) &=& (X\circ c)(t), \,\,\,\,\,\,\,\mbox{for all $t$ in $I$}\\
 c(0) &=& x.
\end{eqnarray*}
Here $I\subset \sR$ is some open  interval of $0$, and $c'(t)$ is
the tangent vector in $T_{c(t)}M$ represented by the curve.

Suppose $x^i$ are local coordinates for $M$, $c^i$ are functions
representing $c$ in these local coordinates, and 
   $X=X^i \frac{\partial}{\partial x^i}$. Then the condition on $c$
is 
$$
   \frac{dc^i}{dt}(t) = X^i\circ c(t), \quad \mbox{for all $t$}.
$$
%%%%%
%%%%%
\end{document}
