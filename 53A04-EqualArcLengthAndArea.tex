\documentclass[12pt]{article}
\usepackage{pmmeta}
\pmcanonicalname{EqualArcLengthAndArea}
\pmcreated{2013-03-22 19:13:36}
\pmmodified{2013-03-22 19:13:36}
\pmowner{pahio}{2872}
\pmmodifier{pahio}{2872}
\pmtitle{equal arc length and area}
\pmrecord{8}{42149}
\pmprivacy{1}
\pmauthor{pahio}{2872}
\pmtype{Example}
\pmcomment{trigger rebuild}
\pmclassification{msc}{53A04}
\pmclassification{msc}{34A34}
\pmclassification{msc}{34A05}
\pmclassification{msc}{26A09}
\pmsynonym{equal area and arc length}{EqualArcLengthAndArea}
\pmrelated{Arcosh}
\pmrelated{HyperbolicFunctions}
\pmrelated{ChainCurve}

% this is the default PlanetMath preamble.  as your knowledge
% of TeX increases, you will probably want to edit this, but
% it should be fine as is for beginners.

% almost certainly you want these
\usepackage{amssymb}
\usepackage{amsmath}
\usepackage{amsfonts}

% used for TeXing text within eps files
%\usepackage{psfrag}
% need this for including graphics (\includegraphics)
%\usepackage{graphicx}
% for neatly defining theorems and propositions
 \usepackage{amsthm}
% making logically defined graphics
%%%\usepackage{xypic}
\usepackage{pstricks}
\usepackage{pst-plot}

% there are many more packages, add them here as you need them

% define commands here
\DeclareMathOperator{\arcosh}{arcosh}
\theoremstyle{definition}
\newtheorem*{thmplain}{Theorem}

\begin{document}
\PMlinkescapeword{ln}

We want to determine the nonnegative differentiable real functions \,$x \mapsto y$\, whose graph has the property that the arc length between any two points of it is the same as the \PMlinkname{area}{AreaOfPlaneRegion} \PMlinkescapetext{bounded} by the curve, the $x$-axis and the ordinate lines of those points.\\

The requirement leads to the equation
\begin{align}
\int_a^x\!\sqrt{1+\left(\frac{dy}{dx}\right)^2}\,dx \;=\; \int_a^x\!y\,dx.
\end{align}
By the fundamental theorem of calculus, we infer from (1) the differential equation
\begin{align}
\sqrt{1+\left(\frac{dy}{dx}\right)^2} \;=\; y,
\end{align}
whence\, $\frac{dy}{dx} = \sqrt{y^2\!-\!1}$.\, In the case\, $y \not\equiv 1$,\, the separation of variables yields
$$\int\!dx \;=\; \int\!\frac{dy}{\sqrt{y^2\!-\!1}},$$
i.e. 
$$x\!+\!C \;=\; \arcosh{y}.$$
Consequently, the equation (2) has the general solution
\begin{align}
y \;=\; \cosh(x\!+\!C)
\end{align}
and the singular solution
\begin{align}
y \;\equiv\; 1.
\end{align}

The functions defined by (3) and (4) are the only \PMlinkescapetext{function types} satisfying the given requirement.\, The graphs are a chain curve (which may be translated in the horizontal direction) and a line parallel to the $x$-axis.\, Evidently, the line is the envelope of the integral curves given be the general solution.
%%%%%
%%%%%
\end{document}
