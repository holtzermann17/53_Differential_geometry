\documentclass[12pt]{article}
\usepackage{pmmeta}
\pmcanonicalname{GeodesicCompleteness}
\pmcreated{2013-06-03 13:04:01}
\pmmodified{2013-06-03 13:04:01}
\pmowner{jacou}{1000048}
\pmmodifier{unlord}{1}
\pmtitle{geodesic completeness}
\pmrecord{14}{42594}
\pmprivacy{1}
\pmauthor{jacou}{1}
\pmtype{Definition}
\pmcomment{trigger rebuild}
\pmclassification{msc}{53C22}
%\pmkeywords{Hopf-Rinow Theorem}

% this is the default PlanetMath preamble.  as your knowledge
% of TeX increases, you will probably want to edit this, but
% it should be fine as is for beginners.

% almost certainly you want these
\usepackage{amssymb}
\usepackage{amsmath}
\usepackage{amsfonts}

% need this for including graphics (\includegraphics)
\usepackage{graphicx}
% for neatly defining theorems and propositions
\usepackage{amsthm}

% making logically defined graphics
%\usepackage{xypic}
% used for TeXing text within eps files
%\usepackage{psfrag}

% there are many more packages, add them here as you need them

% define commands here

\begin{document}
A Riemannian metric on a manifold $M$ is said to be {\bf geodesically complete} iff its geodesic flow is a complete flow,
i.e. iff for every point $p\in M$ and every tangent vector $v\in T_pM$ at $p$ the solution to the geodesic equation
$$
\nabla_{\dot{\gamma}}\dot{\gamma}=0
$$
with initial condition $\gamma(0)=p$, $\dot{\gamma}(0)=v$ is defined for all time. 
The Hopf-Rinow theorem
asserts that a Riemannian metric is complete if and only if the corresponding metric on $M$ defined by
$$
   d(p,q)\colon=\inf\{L(c), \; c\colon[0,1]\to M,\; c(0)=p,\; c(1)=q\}
$$
is a complete metric (i.e. Cauchy sequences converge). Here $L(c)$ denote the length of the smooth curve $c$, i.e.
$$
   L(c)\colon=\int_0^1 \|\dot{c}(t)\|_{c(t)} \,dt
$$
For a proof of the Hopf-Rinow theorem see Milnor's monograph {\em Morse Theory} 
Princeton Annals of Math Studies {\bf 51} page 62.


%%%%%
%%%%%
\end{document}
