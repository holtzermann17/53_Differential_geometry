\documentclass[12pt]{article}
\usepackage{pmmeta}
\pmcanonicalname{Germ}
\pmcreated{2013-03-22 17:25:36}
\pmmodified{2013-03-22 17:25:36}
\pmowner{fernsanz}{8869}
\pmmodifier{fernsanz}{8869}
\pmtitle{germ}
\pmrecord{5}{39801}
\pmprivacy{1}
\pmauthor{fernsanz}{8869}
\pmtype{Definition}
\pmcomment{trigger rebuild}
\pmclassification{msc}{53B99}
\pmrelated{TangentSpace}
\pmdefines{Germ}
\pmdefines{function germ.}

\endmetadata

% this is the default PlanetMath preamble.  as your knowledge
% of TeX increases, you will probably want to edit this, but
% it should be fine as is for beginners.

% almost certainly you want these
\usepackage{amssymb}
\usepackage{amsmath}
\usepackage{amsfonts}
\usepackage{amsthm}

% define commands here
\newcommand{\To}{\longrightarrow}

\theoremstyle{definition}
\newtheorem{defn}{Definition}
\theoremstyle{remark}
\newtheorem{rem}{Remark}
\numberwithin{equation}{section}
\begin{document}
\title{Germ}%
\author{Fernando Sanz Gámiz}%

\begin{defn}[Germ]
Let $M$ and $N$ be manifolds and $x \in M$. We consider all smooth
mappings $f: U_f \to N$, where $U_f$ is some open neighborhood of
$x$ in $M$. We define an equivalence relation on the set of mappings
considered, and we put $f \underset{x}{\sim} g$ if there is some
open neighborhood $V$ of $x$ with $f|_V = g|_V$. The equivalence
class of a mapping $f$ is called the \emph{germ of f at x}, denoted
by $\overline{f}$ or, sometimes, $germ_x f$, and we write
$$\overline{f}:(M,x) \to (N,f(x))$$
\end{defn}

\bigskip

\begin{rem}
Germs arise naturally in differential topolgy. It is very convenient
when dealing with derivatives at the point $x$, as every mapping in
a germ will have the same derivative values and properties in $x$,
and hence can be identified for such purposes: every mapping in a
germ gives rise to the same \emph{tangent vector} of $M$ at $x$.
\end{rem}
%%%%%
%%%%%
\end{document}
