\documentclass[12pt]{article}
\usepackage{pmmeta}
\pmcanonicalname{MovingFrame}
\pmcreated{2013-03-22 16:27:01}
\pmmodified{2013-03-22 16:27:01}
\pmowner{mathcam}{2727}
\pmmodifier{mathcam}{2727}
\pmtitle{moving frame}
\pmrecord{6}{38606}
\pmprivacy{1}
\pmauthor{mathcam}{2727}
\pmtype{Definition}
\pmcomment{trigger rebuild}
\pmclassification{msc}{53A04}
\pmsynonym{frame}{MovingFrame}
\pmrelated{TNBFrame}
\pmdefines{frame}
\pmdefines{orthonormal frame}
\pmdefines{parallelizable}

% this is the default PlanetMath preamble.  as your knowledge
% of TeX increases, you will probably want to edit this, but
% it should be fine as is for beginners.

% almost certainly you want these
\usepackage{amssymb}
\usepackage{amsmath}
\usepackage{amsfonts}

% used for TeXing text within eps files
%\usepackage{psfrag}
% need this for including graphics (\includegraphics)
%\usepackage{graphicx}
% for neatly defining theorems and propositions
%\usepackage{amsthm}
% making logically defined graphics
%%%\usepackage{xypic}

% there are many more packages, add them here as you need them

% define commands here

\begin{document}
Let $M$ be a smooth manifold.  A \emph{moving frame} (sometimes just a \emph{frame}) on $M$ is a choice, for every $P\in M$, of a basis for the tangent space $T_pM$ to $M$ at $P$.  More formally (and abstractly), a frame is a (smooth) section of the principal bundle for $\operatorname{GL}_n$ over $M$.

\subsection*{Examples and Remarks}
\begin{itemize}
\item If $M=\mathbb{R}^n$, then any basis of $\mathbb{R}^n$ trivially gives a frame as well.
\item A more interesting example (and perhaps a source for the definition) is when $M=\mathbb{R}^2-\{(0,0)\},$ and we take the vectors $\frac{\partial}{\partial r}$ and $\frac{\partial}{\partial\theta}$ at a point $(r,\theta)$.  Note that this frame cannot be extended to a smooth frame on all of $\mathbb{R}^2$.
\item Similar to the previous example, one can show that the 2-sphere admits no frames.  A manifold which admits a (global) frame is called \emph{parallelizable.}
\item A key example of a frame is the Frenet frame.
\item One places adjective in front of "moving frame" if that adjective pertains to each basis, e.g. an \emph{orthogonal frame} is a frame for which each basis is orthogonal (with respect to a given inner product).  Given any frame, one can always "orthonormalize" it in a smooth manner to provide an orthonormal frame.
\item Frames arise in general relativity as a formalization of the observation that there is no ``preferred'' observer standpoint.
\end{itemize}

\begin{thebibliography}{9}
\bibitem {wikibinom} Wikipedia's \PMlinkexternal{entry on moving frame}{http://en.wikipedia.org/wiki/Moving_frame}
\end{thebibliography}
%%%%%
%%%%%
\end{document}
