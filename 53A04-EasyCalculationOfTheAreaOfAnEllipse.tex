\documentclass[12pt]{article}
\usepackage{pmmeta}
\pmcanonicalname{EasyCalculationOfTheAreaOfAnEllipse}
\pmcreated{2013-03-22 15:44:18}
\pmmodified{2013-03-22 15:44:18}
\pmowner{cvalente}{11260}
\pmmodifier{cvalente}{11260}
\pmtitle{easy calculation of the area of an ellipse}
\pmrecord{7}{37689}
\pmprivacy{1}
\pmauthor{cvalente}{11260}
\pmtype{Definition}
\pmcomment{trigger rebuild}
\pmclassification{msc}{53A04}

\endmetadata

% this is the default PlanetMath preamble.  as your knowledge
% of TeX increases, you will probably want to edit this, but
% it should be fine as is for beginners.

% almost certainly you want these
\usepackage{amssymb}
\usepackage{amsmath}
\usepackage{amsfonts}

% used for TeXing text within eps files
%\usepackage{psfrag}
% need this for including graphics (\includegraphics)
%\usepackage{graphicx}
% for neatly defining theorems and propositions
%\usepackage{amsthm}
% making logically defined graphics
%%%\usepackage{xypic}

% there are many more packages, add them here as you need them

% define commands here
\begin{document}
Consider the unit circle $\left \{ \right (x,y) \in \mathbb{R}^2 : x^2+y^2\le 1\}$. It's a well known fact that the area of this set is $\pi$.

Now consider the following linear transformation $(x,y)\to(u,v)=(ax,by)$.

The determinant of the transformation is $ab$ and the transformed circle is:

 $\left \{ \right (u,v) \in \mathbb{R}^2 : \left (\frac{u}{a} \right )^2 + \left (\frac{v}{b} \right )^2 \le 1\}$  an ellipse of axis $(a,b)$.

Now since the Jacobian of the transformation is constant, the \PMlinkname{change of variables in integral theorem}{ChangeOfVariablesInIntegralOnMathbbRn} allows us to say the area of the transformed set is $ab$ 
times the area of the original set.

Thus, the area of an ellipse is $\pi a b$.
%%%%%
%%%%%
\end{document}
