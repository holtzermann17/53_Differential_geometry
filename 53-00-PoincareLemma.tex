\documentclass[12pt]{article}
\usepackage{pmmeta}
\pmcanonicalname{PoincareLemma}
\pmcreated{2013-03-22 14:06:28}
\pmmodified{2013-03-22 14:06:28}
\pmowner{matte}{1858}
\pmmodifier{matte}{1858}
\pmtitle{Poincar\'e lemma}
\pmrecord{12}{35509}
\pmprivacy{1}
\pmauthor{matte}{1858}
\pmtype{Theorem}
\pmcomment{trigger rebuild}
\pmclassification{msc}{53-00}
\pmrelated{ExactDifferentialForm}
\pmrelated{ClosedDifferentialFormsOnASimpleConnectedDomain}
\pmrelated{LaminarField}

\endmetadata

% this is the default PlanetMath preamble.  as your knowledge
% of TeX increases, you will probably want to edit this, but
% it should be fine as is for beginners.

% almost certainly you want these
\usepackage{amssymb}
\usepackage{amsthm}
\usepackage{amsmath}
\usepackage{amsfonts}

\newtheorem*{theorem*}{Theorem}

% used for TeXing text within eps files
%\usepackage{psfrag}
% need this for including graphics (\includegraphics)
%\usepackage{graphicx}
% for neatly defining theorems and propositions
\usepackage{amsthm}
% making logically defined graphics
%%%\usepackage{xypic}

% there are many more packages, add them here as you need them

% define commands here

\newcommand{\sR}[0]{\mathbb{R}}
\newcommand{\sC}[0]{\mathbb{C}}
\newcommand{\sN}[0]{\mathbb{N}}
\newcommand{\sZ}[0]{\mathbb{Z}}

\newcommand*{\norm}[1]{\lVert #1 \rVert}
\newcommand*{\abs}[1]{| #1 |}
\begin{document}
\PMlinkescapeword{name}
\PMlinkescapeword{states}
\PMlinkescapeword{measure}
\PMlinkescapeword{degree}

The Poincar\'e lemma states that every closed differential form
is locally \PMlinkname{exact}{ExactDifferentialForm}. 

\begin{theorem*} (Poincar\'e Lemma)
\cite{conlon} Suppose $X$ is a smooth
manifold, $\Omega^k(X)$ is the set of smooth differential
$k$-forms on $X$, and suppose  $\omega$ is a closed form 
in $\Omega^k(X)$ for some $k>0$.
\begin{itemize}
\item
Then for every $x\in X$ there is a neighbourhood $U\subset X$, and a
$(k-1)$-form $\eta\in \Omega^{k-1}(U)$, such that
$$ d\eta = \iota^\ast \omega,$$
where $\iota$ is the inclusion $\iota:U\hookrightarrow X$.
\item If $X$ is contractible, this $\eta$ exists globally; there exists a
$(k-1)$-form $\eta\in \Omega^{k-1}(X)$ such that
$$ d\eta = \omega.$$
\end{itemize}
\end{theorem*}

\subsubsection*{Notes}
Despite the name, the Poincar\'e lemma is an
extremely important result. For instance, in algebraic topology,
the definition of the $k$th de Rham cohomology group
$$ 
  H^k(X) = \frac{ \operatorname{Ker}\{ d\colon \Omega^k(X)\to \Omega^{k+1}(X)\}}{  \operatorname{Im}\{ d\colon \Omega^{k-1}(X)\to \Omega^{k}(X)\}}
$$
can be seen as a measure of the degree in which the Poincar\'e lemma fails.
If $H^k(X)=0$, then every $k$ form is exact, but if $H^k(X)$ is non-zero, then
$X$ has a non-trivial topology (or ``holes'') such that $k$-forms are not
globally exact. For instance, in $X=\sR^2\setminus\{0\}$ with polar coordinates $(r,\phi)$,
the $1$-form $\omega=d\phi$ is not globally exact.


\begin{thebibliography}{9}
 \bibitem {conlon} L. Conlon, \emph{Differentiable Manifolds: A first course},
Birkh\"auser, 1993.
\end{thebibliography}
%%%%%
%%%%%
\end{document}
