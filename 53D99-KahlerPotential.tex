\documentclass[12pt]{article}
\usepackage{pmmeta}
\pmcanonicalname{KahlerPotential}
\pmcreated{2013-03-22 16:33:17}
\pmmodified{2013-03-22 16:33:17}
\pmowner{rspuzio}{6075}
\pmmodifier{rspuzio}{6075}
\pmtitle{K\"ahler potential}
\pmrecord{7}{38740}
\pmprivacy{1}
\pmauthor{rspuzio}{6075}
\pmtype{Definition}
\pmcomment{trigger rebuild}
\pmclassification{msc}{53D99}
\pmsynonym{Kahler potential}{KahlerPotential}

% this is the default PlanetMath preamble.  as your knowledge
% of TeX increases, you will probably want to edit this, but
% it should be fine as is for beginners.

% almost certainly you want these
\usepackage{amssymb}
\usepackage{amsmath}
\usepackage{amsfonts}

% used for TeXing text within eps files
%\usepackage{psfrag}
% need this for including graphics (\includegraphics)
%\usepackage{graphicx}
% for neatly defining theorems and propositions
%\usepackage{amsthm}
% making logically defined graphics
%%%\usepackage{xypic}

% there are many more packages, add them here as you need them

% define commands here

\begin{document}
A {K\"ahler potential} is a real-valued function $f$ defined on some coordinate patch of a  
Hermitean manifold such that the metric of the manifold is given by the expression
\[
 g_{ij*} = {\partial^2 f \over dz^i d{\overline z}^j} .
\]
It turns out that, for every K\'ahler manifold, there will exist a coordinate neighborhood of any 
given point in which the metric can be expresses in terms of a potential this way.

As an elementary example of a K\"ahler potential, we may consider $f(z,{\overline z}) = z
{\overline z}$.  This potential gives rise to the flat metric $ds^2 = dz d{\overline z}$.

K\"ahler potentials have applications in physics. For example, this function $f(x) = \log(x) + g(x)$ relates to the motion of certain subatomic particles called gauginos.

\begin{thebibliography}{1}
\bibitem{tb} T. Barreiro, B. de Carlos \& E. J. Copeland, ``On non-perturbative corrections to the K\"ahler potential'' {\it Physical Review} D57 (1998): 7354 - 7360
\end{thebibliography}
%%%%%
%%%%%
\end{document}
