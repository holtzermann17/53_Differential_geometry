\documentclass[12pt]{article}
\usepackage{pmmeta}
\pmcanonicalname{FundamentalConceptsInDifferentialGeometry}
\pmcreated{2013-03-22 14:47:23}
\pmmodified{2013-03-22 14:47:23}
\pmowner{rspuzio}{6075}
\pmmodifier{rspuzio}{6075}
\pmtitle{fundamental concepts in differential geometry}
\pmrecord{15}{36441}
\pmprivacy{1}
\pmauthor{rspuzio}{6075}
\pmtype{Topic}
\pmcomment{trigger rebuild}
\pmclassification{msc}{53-00}

\endmetadata

% this is the default PlanetMath preamble.  as your knowledge
% of TeX increases, you will probably want to edit this, but
% it should be fine as is for beginners.

% almost certainly you want these
\usepackage{amssymb}
\usepackage{amsmath}
\usepackage{amsfonts}

% used for TeXing text within eps files
%\usepackage{psfrag}
% need this for including graphics (\includegraphics)
%\usepackage{graphicx}
% for neatly defining theorems and propositions
%\usepackage{amsthm}
% making logically defined graphics
%%%\usepackage{xypic}

% there are many more packages, add them here as you need them

% define commands here
\begin{document}
The following is an index of fundamental concepts in differental geometry.  It only deals with basic concepts which are common to all branches of differential geometry.  For concepts pertinent to specific branches of differential geometry, please see concepts in symplectic geometry and concepts in Riemannian geometry 

\subsection{Manifolds}

\begin{itemize}
\item manifold
\item smooth manifold
\item manifold with boundary
\item boundary manifold
\item Riemannian manifold
\item chart
\item coordinate function
\item coordinate map
\item coordinate neighborhood
\item coordinate system
\item submanifold
\item immersion
\item differential structure
\item diffeomorphism
\end{itemize}

\subsection{Vector and Tensor Fields}
\begin{itemize}
\item vector field
\item tensor product
\item contraction
\item exterior product
\item differential form
\item exterior derivative
\item tensor field
\item jet
\item spinor field
\item Lie bracket
\item Lie derivative
\item flow
\item integrable
\item Pfaffian system
\item frame fields
\item Lie groups and algebras 
\item fields as section on bundles
\end{itemize}

\subsection{Bundles and Connections}
\begin{itemize}
\item fibre bundle
\item vector bundle
\item tangent bundle
\item cotangent bundle
\item jet bundle
\item connection
\item affine connection
\item Riemannian metric
\item curvature
\item torsion
\item geodesic
\item homotopic classification and characteristic classes
\item space of connections
\end{itemize}

\subsection{Gauge theory}
\begin{itemize}
\item non-Abelian gauge theory
\end{itemize}


%%%%%
%%%%%
\end{document}
