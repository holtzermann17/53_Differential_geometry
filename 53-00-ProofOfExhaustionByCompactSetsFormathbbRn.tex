\documentclass[12pt]{article}
\usepackage{pmmeta}
\pmcanonicalname{ProofOfExhaustionByCompactSetsFormathbbRn}
\pmcreated{2013-03-22 15:51:38}
\pmmodified{2013-03-22 15:51:38}
\pmowner{cvalente}{11260}
\pmmodifier{cvalente}{11260}
\pmtitle{proof of exhaustion by compact sets for $\mathbb{R}^n$}
\pmrecord{5}{37850}
\pmprivacy{1}
\pmauthor{cvalente}{11260}
\pmtype{Proof}
\pmcomment{trigger rebuild}
\pmclassification{msc}{53-00}

\endmetadata

% this is the default PlanetMath preamble.  as your knowledge
% of TeX increases, you will probably want to edit this, but
% it should be fine as is for beginners.

% almost certainly you want these
\usepackage{amssymb}
\usepackage{amsmath}
\usepackage{amsfonts}

% used for TeXing text within eps files
%\usepackage{psfrag}
% need this for including graphics (\includegraphics)
%\usepackage{graphicx}
% for neatly defining theorems and propositions
%\usepackage{amsthm}
% making logically defined graphics
%%%\usepackage{xypic}

% there are many more packages, add them here as you need them

% define commands here
\newcommand{\inter}{\operatorname{int}}
\begin{document}
First consider $A\subset \mathbb{R}^n$ to be a bounded open set and designate the open ball centered at $x$ with radius $r$ by $B_r(x)$

Construct $C_n=\bigcup_{x \in \partial A} B_{\frac{1}{n}}(x)$, where $\partial A$ is the boundary of $A$ and define $K_n = A \backslash C_n$.

\begin{itemize}
\item $K_n$ is compact.

It is bounded since $K_n \subset A$ and $A$ is by assumption bounded.
 $K_n$ is also closed. To see this consider $x \in \partial K_n$ but $x \notin K_n$. Then there exists $y \in \partial A$ and $0<r<\frac{1}{n}$ such that $x \in B_r(y)$.
But $B_r(y) \cap K_n = \{\}$ because $B_{\frac{1}{n}}(y) \cap K_n = \{\}$ and $0<r<\frac{1}{n} \implies B_r(y) \subset B_{\frac{1}{n}}(y)$.
This implies that $x \notin \partial K_n$ and we have a contradiction. $K_n$ is therefore closed.

\item $K_{n} \subset \inter K_{n+1}$

Suppose $x \in K_n$ and $x \notin \inter K_{n+1}$. This means that for all $y \in \partial A$, $x \in \overline{B_{\frac{1}{n+1}}(y)} \vee x \in \mathbb{R}^n \backslash A$.
Since $x \in K_n \implies x \in A$ we must have $x \in \overline{B_{\frac{1}{n+1}}(y)}$.
But $x \in \overline{B_{\frac{1}{n+1}}(y)} \subset B_{\frac{1}{n}}(y) \implies x \notin K_n$ and we have a contradiction.

\item $\bigcup_{n=1}^{\infty} K_n = A$

Suppose $x \in A$, since $A$ is open there must exist $r>0$ such that $B_r(x) \subset A$.
Considering $n$ such that $\frac{1}{n}<r$ we have that $x \notin B_{\frac{1}{n}}(y)$ for all $y \in \partial A$ and thus $x \in K_n$.

\end{itemize}

Finally if $A$ is not bounded consider $A_k = A \cap B_{k}(0)$ and define $K_n = \bigcup_{k=1}^n K_{k,n}$ where $K_{k,n}$ is the set resulting from the previous construction on the bounded set $A_k$.

\begin{itemize}

\item $K_n$  will be compact because it is the finite union of compact sets.

\item $K_{n} \subset \inter K_{n+1}$ because $K_{k,n} \subset \inter K_{k,n+1}$ and $\inter(A \cup B) \subset \inter A \cup \inter B$

\item $\bigcup_{n=1}^{\infty} K_n = A$

First find $k$ such that $x \in A_k$.
This will always be possible since all it requires is that $k>|x|$.
Finally since $n > k \implies K_{k,n} \subset K_{n,n}$ by construction the argument for the bounded case is directly applicable.

\end{itemize}



\end{document}
%%%%%
%%%%%
\end{document}
