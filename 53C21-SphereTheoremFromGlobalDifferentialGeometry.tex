\documentclass[12pt]{article}
\usepackage{pmmeta}
\pmcanonicalname{SphereTheoremFromGlobalDifferentialGeometry}
\pmcreated{2013-03-22 15:54:09}
\pmmodified{2013-03-22 15:54:09}
\pmowner{juanman}{12619}
\pmmodifier{juanman}{12619}
\pmtitle{sphere theorem from global differential geometry}
\pmrecord{8}{37905}
\pmprivacy{1}
\pmauthor{juanman}{12619}
\pmtype{Theorem}
\pmcomment{trigger rebuild}
\pmclassification{msc}{53C21}
\pmclassification{msc}{53C20}
%\pmkeywords{sectional curvature}
\pmrelated{Curvature}
\pmrelated{Connection}

% this is the default PlanetMath preamble.  as your knowledge
% of TeX increases, you will probably want to edit this, but
% it should be fine as is for beginners.

% almost certainly you want these
\usepackage{amssymb}
\usepackage{amsmath}
\usepackage{amsfonts}

% used for TeXing text within eps files
%\usepackage{psfrag}
% need this for including graphics (\includegraphics)
%\usepackage{graphicx}
% for neatly defining theorems and propositions
%\usepackage{amsthm}
% making logically defined graphics
%%%\usepackage{xypic}

% there are many more packages, add them here as you need them

% define commands here

\begin{document}
This theorem, as do Carmo refers it, is one of the most beautiful theorems in Riemannian geometry:

{\bf sphere theorem.} {\it Let $M$ be a n-dimensional compact simply connected Riemannian manifold, whose sectional curvature $K$ satisfies
$$0<K_{\rm max}/4<K\le K_{\rm max}$$
Then $M$ is homeomorphic to a sphere.}

\begin{thebibliography}{9}
\bibitem{mpdc} M. P. do Carmo, {\em Riemannian Geometry}, Birkh\"auser, Boston, 1992.
\end{thebibliography}




%%%%%
%%%%%
\end{document}
