\documentclass[12pt]{article}
\usepackage{pmmeta}
\pmcanonicalname{CircleOfCurvature}
\pmcreated{2013-03-22 16:59:46}
\pmmodified{2013-03-22 16:59:46}
\pmowner{rspuzio}{6075}
\pmmodifier{rspuzio}{6075}
\pmtitle{circle of curvature}
\pmrecord{13}{39277}
\pmprivacy{1}
\pmauthor{rspuzio}{6075}
\pmtype{Definition}
\pmcomment{trigger rebuild}
\pmclassification{msc}{53A04}
\pmsynonym{osculating circle}{CircleOfCurvature}
\pmrelated{CurvatureOfaCircle}
\pmrelated{OsculatingCurve}
\pmdefines{radius of curvature}
\pmdefines{center of curvature}
\pmdefines{centre of curvature}

\endmetadata

% this is the default PlanetMath preamble.  as your knowledge
% of TeX increases, you will probably want to edit this, but
% it should be fine as is for beginners.

% almost certainly you want these
\usepackage{amssymb}
\usepackage{amsmath}
\usepackage{amsfonts}

% used for TeXing text within eps files
%\usepackage{psfrag}
% need this for including graphics (\includegraphics)
%\usepackage{graphicx}
% for neatly defining theorems and propositions
 \usepackage{amsthm}
% making logically defined graphics
%%%\usepackage{xypic}

% there are many more packages, add them here as you need them

% define commands here

\theoremstyle{definition}
\newtheorem*{thmplain}{Theorem}

\begin{document}
Let a given plane curve $\gamma$ have a definite 
\PMlinkname{curvature}{CurvaturePlaneCurve} $\kappa$ in 
a point $P$ of $\gamma$.\, The {\em circle of curvature} of $\gamma$ in the 
point $P$ is the circle which has the radius $\frac{1}{|\kappa|}$ 
and which has with the curve the common tangent in $P$ and which 
in a neighbourhood of $P$ is on the same side as the curve.

The radius $\varrho$ of the circle of curvature is the 
{\em radius of curvature} of $\gamma$ in $P$.\, The center of the 
circle of curvature is the {\em center of curvature} of $\gamma$ in $P$.

When the curve $\gamma$ is given in the parametric form
              $$x = x(t),\;\; y ´= y(t),$$
the coordinates of the center of curvature belonging to the point\, $(x,\,y)$\, 
of the curve are
$$\xi = x-\frac{(x'^2+y'^2)y'}{x'y''-x''y'},\quad
\eta = y+\frac{(x'^2+y'^2)x'}{x'y''-x''y'}.$$


\textbf{Example.}\, Since the curvature of the parabola \,$y = x^2$\, 
in the origin is\; $-2$,\, the corresponding radius of curvature is 
$\frac{1}{2}$ and the center of curvature\, $(0,\,\frac{1}{2})$.

Furthermore, it is possible to define the circle of curvature without
first knowing about curvature of the curve.  (In fact, using this definition,
one could reverse the procedure and define curvature as the radius of
the circle of curvature.)   We may define the circle of curvature 
a point $P$ of $\gamma$ as the unique circle passing through $P$ which 
makes a second-order contact with $\gamma$ at $P$.  
%%%%%
%%%%%
\end{document}
