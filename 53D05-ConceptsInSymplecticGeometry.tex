\documentclass[12pt]{article}
\usepackage{pmmeta}
\pmcanonicalname{ConceptsInSymplecticGeometry}
\pmcreated{2013-03-22 14:46:37}
\pmmodified{2013-03-22 14:46:37}
\pmowner{matte}{1858}
\pmmodifier{matte}{1858}
\pmtitle{concepts in symplectic geometry}
\pmrecord{8}{36423}
\pmprivacy{1}
\pmauthor{matte}{1858}
\pmtype{Topic}
\pmcomment{trigger rebuild}
\pmclassification{msc}{53D05}

\endmetadata

% this is the default PlanetMath preamble.  as your knowledge
% of TeX increases, you will probably want to edit this, but
% it should be fine as is for beginners.

% almost certainly you want these
\usepackage{amssymb}
\usepackage{amsmath}
\usepackage{amsfonts}
\usepackage{amsthm}

\usepackage{mathrsfs}

% used for TeXing text within eps files
%\usepackage{psfrag}
% need this for including graphics (\includegraphics)
%\usepackage{graphicx}
% for neatly defining theorems and propositions
%
% making logically defined graphics
%%%\usepackage{xypic}

% there are many more packages, add them here as you need them

% define commands here

\newcommand{\sR}[0]{\mathbb{R}}
\newcommand{\sC}[0]{\mathbb{C}}
\newcommand{\sN}[0]{\mathbb{N}}
\newcommand{\sZ}[0]{\mathbb{Z}}

 \usepackage{bbm}
 \newcommand{\Z}{\mathbbmss{Z}}
 \newcommand{\C}{\mathbbmss{C}}
 \newcommand{\R}{\mathbbmss{R}}
 \newcommand{\Q}{\mathbbmss{Q}}



\newcommand*{\norm}[1]{\lVert #1 \rVert}
\newcommand*{\abs}[1]{| #1 |}



\newtheorem{thm}{Theorem}
\newtheorem{defn}{Definition}
\newtheorem{prop}{Proposition}
\newtheorem{lemma}{Lemma}
\newtheorem{cor}{Corollary}
\begin{document}
\subsubsection*{Linear theory}
\begin{enumerate}
\item symplectic vector space, symplectic matrix
\item symplectic complement
\end{enumerate}

\subsubsection*{Symplectic manifolds}
\begin{enumerate}
\item symplectic manifold (\PMlinkname{examples}{ExamplesOfSymplecticManifolds})
\item Lagrangian submanifold, isotropic submanifold
\item symplectic vector field
\item symplectomorphism, canonical transformation
\item Hamilton equations
\item \PMlinkname{Darboux's Theorem}{DarbouxsTheoremSymplecticGeometry}
\item Poisson bracket
\item Moser's theorem
\item Gray stability theorem
\item momentum map
\end{enumerate}
%%%%%
%%%%%
\end{document}
