\documentclass[12pt]{article}
\usepackage{pmmeta}
\pmcanonicalname{CurvatureOfACircle}
\pmcreated{2013-03-22 15:50:30}
\pmmodified{2013-03-22 15:50:30}
\pmowner{cvalente}{11260}
\pmmodifier{cvalente}{11260}
\pmtitle{curvature of a circle}
\pmrecord{9}{37820}
\pmprivacy{1}
\pmauthor{cvalente}{11260}
\pmtype{Example}
\pmcomment{trigger rebuild}
\pmclassification{msc}{53A04}
\pmrelated{circle}
\pmrelated{curvature}
\pmrelated{Connection}
\pmrelated{CircleOfCurvature}

% this is the default PlanetMath preamble.  as your knowledge
% of TeX increases, you will probably want to edit this, but
% it should be fine as is for beginners.

% almost certainly you want these
\usepackage{amssymb}
\usepackage{amsmath}
\usepackage{amsfonts}

% used for TeXing text within eps files
%\usepackage{psfrag}
% need this for including graphics (\includegraphics)
%\usepackage{graphicx}
% for neatly defining theorems and propositions
%\usepackage{amsthm}
% making logically defined graphics
%%%\usepackage{xypic}

% there are many more packages, add them here as you need them

% define commands here
\begin{document}
Let $C_r$ be a circle of radius $r$ centered at the origin.

A canonical parameterization of the curve is (counterclockwise)

$$ g(s) = r\left( \cos \left( \frac{s}{r} \right), \sin \left( \frac{s}{r}\right) \right) $$

for $s \in (0, 2\pi r)$ (actually this leaves out the point $(r,0)$ but this could be treated via another parameterization taking $s \in (-\pi r, \pi r)$)

Differentiating the parameterization we get

$$ \mathbf{T} = g'(s) = \left( -\sin \left( \frac{s}{r} \right), \cos \left( \frac{s}{r}\right) \right)$$

and this results in the normal

$$\mathbf{N} = J \cdot\mathbf{T} = -\left(\cos\left(\frac{s}{r}\right),\;\sin\left(\frac{s}{r}\right)\right) 
= -\frac{g(s)}{r}$$

Differentiating $g$ a second time we can calculate the curvature

$$\mathbf{T}' = -\frac{1}{r}\left(\cos\left(\frac{s}{r}\right),\;\sin\left(\frac{s}{r}\right)\right) =  \frac{1}{r} \mathbf{N}$$

and by definition

$$ \mathbf{T}' = k\mathbf{N}\;\; \therefore\; k = \frac{1}{r} $$

and thus the curvature of a circle of radius $r$ is $\displaystyle{\frac{1}{r}}$ provided that the positive direction on the circle is anticlockwise; otherwise it is $\displaystyle{-\frac{1}{r}}$.
%%%%%
%%%%%
\end{document}
