\documentclass[12pt]{article}
\usepackage{pmmeta}
\pmcanonicalname{Geodesic}
\pmcreated{2013-03-22 14:06:37}
\pmmodified{2013-03-22 14:06:37}
\pmowner{Mathprof}{13753}
\pmmodifier{Mathprof}{13753}
\pmtitle{geodesic}
\pmrecord{22}{35513}
\pmprivacy{1}
\pmauthor{Mathprof}{13753}
\pmtype{Definition}
\pmcomment{trigger rebuild}
\pmclassification{msc}{53C22}
%\pmkeywords{shortest path}
\pmrelated{connection}
\pmrelated{Connection}
\pmdefines{focal point}
\pmdefines{minimizing geodesic}
\pmdefines{geodesic curve}

% this is the default PlanetMath preamble.  as your knowledge
% of TeX increases, you will probably want to edit this, but
% it should be fine as is for beginners.

% almost certainly you want these
\usepackage{amssymb}
\usepackage{amsmath}
\usepackage{amsfonts}

% used for TeXing text within eps files
%\usepackage{psfrag}
% need this for including graphics (\includegraphics)
%\usepackage{graphicx}
% for neatly defining theorems and propositions
\usepackage{amsthm}
% making logically defined graphics
%%%\usepackage{xypic}

% there are many more packages, add them here as you need them

% define commands here


\DeclareMathOperator{\Length}{Length}
\begin{document}
\PMlinkescapeword{straight}

Let $M$ be a differentiable manifold (at least two times differentiable) with affine connection $\nabla$. The solution to the equation \[\nabla_{\dot\gamma}\dot{\gamma}=0\] defined in the interval $[0,a]$, is called a \emph{geodesic} or a \emph{geodesic curve}. 
It can be shown that if $\nabla$ is a Levi-Civita connection and $a$ is `small enough', then the curve $\gamma$ is the shortest possible curve between the points $\gamma(0)$ and $\gamma(a)$, and is often referred to as a \emph{minimizing geodesic} between these points.

Conversely, any curve which minimizes the \PMlinkescapetext{distance} between two arbitrary points in a manifold, is a geodesic.

\PMlinkescapetext{Simple} examples of geodesics includes straight lines in Euclidean space ($\mathbb{R}^n$) and great circles on spheres (such as the equator of earth).  The latter of which is not minimizing if the geodesic from the point $p$ is extended beyond its antipodal point. This example also points out to us that between any two points there may be more than one geodesic. In fact, between a point and its antipodal point on the sphere, there are an infinite number of geodesics. Given a \PMlinkescapetext{fixed point} $p$, it is also a property for a point $q$ (known as a \emph{focal point} of $p$) where different geodesics issuing from $p$ intersects, to be the point where any given geodesic from $p$ ceases to be minimizing. 

\paragraph{Coordinates}
In coordinates the equation is given by the system \[\frac{d^2x_k}{dt^2}+\sum_{i,j}\Gamma^k_{ij}\frac{dx_i}{dt}\frac{dx_j}{dt}=0 \qquad 1\leq k \leq n\]
where $\Gamma^k_{ij}$ is the Christoffel symbols (see entry about connection), $t$ is the parameter of the curve and $\{x_1, \ldots , x_n\}$ are coordinates on $M$.

The formula follows since if $\displaystyle {\dot{\gamma}}= \sum_i\frac{dx_i}{dt}\partial_{x_i}$, where  $\{\partial_{x_1}, \ldots , \partial_{x_n}\}$ are the corresponding coordinate vectors, we have 
\begin{align*}
\nabla_{\dot{\gamma}}{\dot{\gamma}}&= \nabla_{\sum_i\frac{dx_i}{dt}\partial_{x_i}}{\sum_j\frac{dx_j}{dt}\partial_{x_j}}\\
&=\sum_k\dot{\gamma}\left(\frac{dx_k}{dt}\right)\partial_{x_k}+ \sum_{i,j}\frac{dx_j}{dt}\frac{dx_i}{dt}\nabla_{\partial_{x_i}}\partial_{x_j} \\
&=\sum_k\left( \frac{d^2x_k}{dt^2}+\sum_{i,j} \frac{dx_i}{dt}\frac{dx_j}{dt}\Gamma^k_{ij}\right)\partial_{x_k}.
\end{align*}

\paragraph{Metric spaces}
A geodesic in a metric space $(X,d)$ is simply a continuous $f:[0,a]\to X$ such that the 
\PMlinkname{length}{LengthOfCurveInAMetricSpace} of $f$ is $a$. 
Of course, the \PMlinkescapetext{length} may be infinite. A geodesic metric space  is a metric space 
where the distance between two points may be realized by a geodesic.

%%%%%
%%%%%
\end{document}
