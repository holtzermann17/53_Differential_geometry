\documentclass[12pt]{article}
\usepackage{pmmeta}
\pmcanonicalname{GermSpace}
\pmcreated{2013-03-22 19:18:20}
\pmmodified{2013-03-22 19:18:20}
\pmowner{joking}{16130}
\pmmodifier{joking}{16130}
\pmtitle{germ space}
\pmrecord{4}{42242}
\pmprivacy{1}
\pmauthor{joking}{16130}
\pmtype{Definition}
\pmcomment{trigger rebuild}
\pmclassification{msc}{53B99}

\endmetadata

% this is the default PlanetMath preamble.  as your knowledge
% of TeX increases, you will probably want to edit this, but
% it should be fine as is for beginners.

% almost certainly you want these
\usepackage{amssymb}
\usepackage{amsmath}
\usepackage{amsfonts}

% used for TeXing text within eps files
%\usepackage{psfrag}
% need this for including graphics (\includegraphics)
%\usepackage{graphicx}
% for neatly defining theorems and propositions
%\usepackage{amsthm}
% making logically defined graphics
%%%\usepackage{xypic}

% there are many more packages, add them here as you need them

% define commands here

\begin{document}
Let $X$, $Y$ be topological spaces and $x\in X$. Consider the set of all continuous functions
$$C(X,Y)=\{f:X\to Y\ |\ f\mbox{ is continuous}\}.$$
For any two functions $f,g:X\to Y$ we put
$$f\sim_{x} g$$
if and only if there exists an open neighbourhood $U\subseteq X$ of $x$ such that
$$f_{|U}=g_{|U}.$$
The corresponding quotient set is called \textbf{the germ space} at $x\in X$ and we denote it by $G_x(X,Y)$.

More generally, if $X$, $Y$ are topological spaces with $x\in X$, then consider the following set:
$$C_x(X,Y)=\{f:U\to Y\ |\ f\mbox{ is continuous and }U\mbox{ is an open neighbourhood of }x\}.$$
Again we define a relation on $C_x(X,Y)$. If $f:U\to Y$ and $g:U'\to Y$, then put
$$f\sim_{x} g$$
if and only if there exists and open neighbourhood $V\subseteq X$ of $x$ such that $V\subseteq U\cap U'$ and
$$f_{|V}=g_{|V}.$$
The corresponding set is called \textbf{the generalized germ space} at $x\in X$ and we denote it by $G_x^*(X,Y)$.

Note that if $Y=\mathbb{R}$ or $Y=\mathbb{C}$ (or $Y$ is any topological ring), then both $G_x(X,Y)$ and $G_x^*(X,Y)$ have a well-defined ring structure via pointwise addition and multiplication.
%%%%%
%%%%%
\end{document}
