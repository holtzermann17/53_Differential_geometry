\documentclass[12pt]{article}
\usepackage{pmmeta}
\pmcanonicalname{Cycloid}
\pmcreated{2013-11-29 11:13:47}
\pmmodified{2013-11-29 11:13:47}
\pmowner{matte}{1858}
\pmmodifier{pahio}{2872}
\pmtitle{cycloid}
\pmrecord{19}{37100}
\pmprivacy{1}
\pmauthor{matte}{2872}
\pmtype{Definition}
\pmcomment{trigger rebuild}
\pmclassification{msc}{53A04}
\pmrelated{ArcLength}
\pmrelated{GoniometricFormulae}
\pmrelated{EvoluteOfCycloid}

\usepackage{amssymb}
\usepackage{amsmath}
\usepackage{amsfonts}
\usepackage{amsthm}
\usepackage{pstricks}
\usepackage{pst-plot}

\usepackage{mathrsfs}

\newcommand{\sR}[0]{\mathbb{R}}
\newcommand{\sC}[0]{\mathbb{C}}
\newcommand{\sN}[0]{\mathbb{N}}
\newcommand{\sZ}[0]{\mathbb{Z}}

 \usepackage{bbm}
 \newcommand{\Z}{\mathbbmss{Z}}
 \newcommand{\C}{\mathbbmss{C}}
 \newcommand{\F}{\mathbbmss{F}}
 \newcommand{\R}{\mathbbmss{R}}
 \newcommand{\Q}{\mathbbmss{Q}}



\newcommand*{\norm}[1]{\lVert #1 \rVert}
\newcommand*{\abs}[1]{| #1 |}



\newtheorem{thm}{Theorem}
\newtheorem{defn}{Definition}
\newtheorem{prop}{Proposition}
\newtheorem{lemma}{Lemma}
\newtheorem{cor}{Corollary}
\begin{document}
A \emph{cycloid} is a curve that a point on the perimeter of a wheel \PMlinkescapetext{traces} 
when rolling along the $x$-axis without slipping.\, If the 
radius of the rolling wheel is $a$, then the cycloid may be 
presented in the parametric form
\begin{align}
\begin{cases}
x \;=\; a(\varphi-\sin\varphi),\\
y \;=\; a(1-\cos\varphi),
\end{cases}
\end{align}
where $\varphi$ expresses the angle rotated by the wheel around its \PMlinkescapetext{center}.

In what follows, a blue curve indicates a cycloid (or a portion thereof) and red line segments indicate radii of the wheel.

Below is a picture of the wheel on the $x$-axis with\, $\varphi = 0$.

\begin{center}
\begin{pspicture}(-1,0)(1,2)
\pscircle(0,1){1}
\psline[linecolor=red](0,0)(0,1)
\psline{->}(-1,0)(1,0)
\rput[b](1,-0.3){$x$}
\psdots(0,0)(0,1)
\rput[l](-1,0){.}
\rput[a](0,2){.}
\end{pspicture}
\end{center}

As the wheel rolls, $\varphi$ increases.\, To obtain the cycloid, we keep track of the \PMlinkname{path}{Path} along which the \PMlinkescapetext{fixed point} of the wheel has travelled.

\begin{center}
\begin{pspicture}(-1,0)(3.2,2)
\pscurve[linecolor=blue](0,0)(0.00126,0.019215)(0.01,0.07612)(0.0335,0.16853)(0.0783,0.2929)(0.1503,0.44443)
                        (0.25422,0.61732)(0.39366,0.80491)(0.5708,1)(0.78636,1.19509)(1.04,1.3827)(1.3284,1.5556)
                        (1.45,1.6)
\psarc[linecolor=green](2.2,1){1}{135}{270}
\psarc(2.2,1){1}{-90}{139}
\psline[linecolor=red](2.2,0)(2.2,1)(1.45,1.6)
\psarc(2.2,1){0.15}{135}{270}
\rput[r](2,0.9){$\varphi$}
\psline{-}(-1,0)(0,0)
\psline[linecolor=green](0,0)(2.2,0)
\psline{->}(2.2,0)(3.2,0)
\psdots(2.2,1)(1.45,1.6)
\rput[b](3.2,-0.3){$x$}
\rput[l](-1,0){.}
\rput[a](2.2,2){.}
\end{pspicture}
\end{center}

After the wheel has completed a full turn, the cycloid takes a sharp turn due to the fact that the point hits the 
$x$-axis, then begins travelling upwards again.

Thus, below is the graph of a cycloid for\, $a=1$.

\begin{center}
\begin{pspicture}(-7,-1)(7,3)
\pscurve[linecolor=blue]{-}(-6.5374,0.61732)(-6.4335,0.44443)(-6.3615,0.2929)(-6.317,0.16853)
                            (-6.2932,0.07612)(-6.2844,0.019215)(-6.2832,0)
\pscurve[linecolor=blue](-6.282,0.019215)(-6.2732,0.07612)(-6.25,0.16853)(-6.205,0.2929)(-6.133,0.44443)
                        (-6.03,0.61732)(-5.89,0.80491)(-5.7124,1)(-5.4969,1.19509)(-5.24357,1.3827)(-4.955,1.5556)
                        (-4.6341,1.7071)(-4.2862,1.83147)(-3.917,1.924)(-3.533,1.98)(-3.1416,2)(-2.75,1.98)
                        (-2.36621,1.924)(-2,1.83147)(-1.649,1.7071)(-1.3284,1.5556)(-1.04,1.3827)(-0.78636,1.19509)
                        (-0.5708,1)(-0.39366,0.80491)(-0.25422,0.61732)(-0.1503,0.44443)(-0.0783,0.2929)
                        (-0.0335,0.16853)(-0.01,0.07612)(-0.00126,0.019215)(0,0)
\pscurve[linecolor=blue](0,0)(0.00126,0.019215)(0.01,0.07612)(0.0335,0.16853)(0.0783,0.2929)(0.1503,0.44443)
                        (0.25422,0.61732)(0.39366,0.80491)(0.5708,1)(0.78636,1.19509)(1.04,1.3827)(1.3284,1.5556)
                        (1.649,1.7071)(2,1.83147)(2.36621,1.924)(2.75,1.98)(3.1416,2)(3.533,1.98)(3.917,1.924)
                        (4.2862,1.83147)(4.6341,1.7071)(4.955,1.5556)(5.24357,1.3827)(5.4969,1.19509)(5.7124,1)
                        (5.89,0.80491)(6.03,0.61732)(6.133,0.44443)(6.205,0.2929)(6.25,0.16853)(6.2732,0.07612)
                        (6.282,0.019215)
\pscurve[linecolor=blue]{-}(6.2832,0)(6.2844,0.019215)(6.2932,0.07612)(6.317,0.16853)(6.3615,0.2929)(6.4335,0.44443)
                            (6.5374,0.61732)
\psaxes{->}(0,0)(-6.5,-0.5)(6.5,2.3)
\rput[a](6.6,-0.25){$x$}
\rput[r](-0.22,2.35){$y$}
\rput[l](-6.5,0){.}
\end{pspicture}
\end{center}

The graph of a cycloid for any $a$ can be obtained by replacing 
$1$ with $a$ and $2$ with $2a$ on the $y$-axis of the graph 
above.\\

The slope of the tangent line of the cycloid (1) is
$$\frac{dy}{dx} \;=\; \frac{dy}{d\varphi}:\frac{dx}{d\varphi} 
\;=\; \frac{\sin\varphi}{1-\cos\varphi} 
\;=\; \cot\frac{\varphi}{2},$$
which is not defined when $\varphi$ is a multiple of $2\pi$.\, 
At these points the quotient $\frac{\sin\varphi}{1-\cos\varphi}$ 
has the left limit $-\infty$ and the right limit $+\infty$; this 
means vertical tangent lines and thus the ordinary cusps of the 
continuous curve.\\

The length of one arc of the cycloid formed by one revolution of the circle (\PMlinkname{e.g.}{Eg}\, $0 \le \varphi \le 2\pi$) is
\begin{align*}
\int_0^{2\pi}\!\sqrt{\left(\frac{dx}{d\varphi}\right)^2\!+\!\left(\frac{dy}{d\varphi}\right)^2}\,d\varphi &\;=\; a\!\int_0^{2\pi}\!\sqrt{(1\!-\!\cos\varphi)^2\!+\!(\sin\varphi)^2}\,d\varphi \\
& \;=\; a\!\int_0^{2\pi}\!\sqrt{2(1\!-\!\cos\varphi)}\,d\varphi \\
& \;=\; 2a\!\int_0^{2\pi}\!\sin\frac{\varphi}{2}\,d\varphi.
\end{align*}
Therefore, the length of one arc of the cycloid is $8a$, \PMlinkname{i.e.}{Ie} four times the diameter of the circle.
%%%%%
%%%%%
\end{document}
