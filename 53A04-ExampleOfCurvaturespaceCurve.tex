\documentclass[12pt]{article}
\usepackage{pmmeta}
\pmcanonicalname{ExampleOfCurvaturespaceCurve}
\pmcreated{2013-03-22 15:40:58}
\pmmodified{2013-03-22 15:40:58}
\pmowner{bloftin}{6104}
\pmmodifier{bloftin}{6104}
\pmtitle{example of curvature (space curve)}
\pmrecord{8}{37624}
\pmprivacy{1}
\pmauthor{bloftin}{6104}
\pmtype{Example}
\pmcomment{trigger rebuild}
\pmclassification{msc}{53A04}
\pmrelated{PositionVector}

\endmetadata

% this is the default PlanetMath preamble.  as your knowledge
% of TeX increases, you will probably want to edit this, but
% it should be fine as is for beginners.

% almost certainly you want these
\usepackage{amssymb}
\usepackage{amsmath}
\usepackage{amsfonts}

% used for TeXing text within eps files
%\usepackage{psfrag}
% need this for including graphics (\includegraphics)
\usepackage{graphicx}
% for neatly defining theorems and propositions
%\usepackage{amsthm}
% making logically defined graphics
%%%\usepackage{xypic}

% there are many more packages, add them here as you need them

% define commands here
\begin{document}
Example space curves and calculating their curvatures using the formula

$$\kappa(t) =\frac{\Vert {\bf r}'(t)\times {\bf r}''(t)\Vert}{ \Vert {\bf r}'(t)\Vert^3}$$


{\bf 1.}  ${\bf r}(t) = 3t \hat{i} + t^2\hat{j} - 4t^2\hat{k}$

the first derivative

${\bf r}'(t) = 3 \hat{i} + 2t\hat{j} - 8t\hat{k}$

vector magnitude of the derivative

$\Vert {\bf r}'(t)\Vert = \sqrt{ 3^2 + (2t)^2 + (-8t)^2}$

$\Vert {\bf r}'(t)\Vert = \sqrt{ 9 + 4t^2 + 16t^2} = \sqrt{9 + 20 t^2}$

the second derivative

${\bf r}''(t) = 2 \hat{j} - 8 \hat{k}$

the cross product 

${\bf r}'(t)\times {\bf r}''(t) = \left|
\begin{array}{ccc}
\hat{i} & \hat{j} & \hat{k} \\
3 & 2t & -8t \\
0 & 2 & -8 \\
\end{array}\right| = (-16t + 16t)\hat{i} - (-24)\hat{j} + 6\hat{k}$


${\bf r}'(t)\times {\bf r}''(t) = 24\hat{j} + 6\hat{k}$

$\Vert {\bf r}'(t)\times {\bf r}''(t) \Vert = \sqrt{576 + 36} = \sqrt{612} = 2\sqrt{153}$

$\Vert {\bf r}'(t)\Vert^3 = (9 + 20t^2)^{3/2}$

$\kappa(t) = \frac{2\sqrt{153}}{(9 + 20t^2)^{3/2}}$

{\bf 2.} Calculate the curvature of the right circular helix as given in the plot below and defined as 

${\bf r}(t) = \cos t \hat{i} + \sin t \hat{j} + t\hat{k}$

\begin{figure}
\includegraphics[scale=.6]{helix.eps}
\vspace{10 pt}
\end{figure}

${\bf r}'(t) = -\sin t \hat{i} + \cos t\hat{j} + \hat{k}$

$\Vert {\bf r}'(t)\Vert = \sqrt{ \sin^2 t + \cos^2 t + 1^2} = \sqrt{2}$

${\bf r}''(t) = -\cos t \hat{i} -\sin t \hat{j}$

${\bf r}'(t)\times {\bf r}''(t) = \left|
\begin{array}{ccc}
\hat{i} & \hat{j} & \hat{k} \\
-\sin t & \cos t & 1 \\
-\cos t & -\sin t & 0 \\
\end{array}\right| = \sin t\hat{i} - \cos t\hat{j} + (\sin^2 t + \cos^2 t)\hat{k}$


${\bf r}'(t)\times {\bf r}''(t) =  \sin t\hat{i} - \cos t\hat{j} + \hat{k}$

$\Vert {\bf r}'(t)\times {\bf r}''(t) \Vert = \sqrt{\sin^2 t + \cos^2 t + 1^2} = \sqrt{2}$

$\Vert {\bf r}'(t)\Vert^3 = 2^{3/2}$

$\kappa(t) = \frac{\sqrt{2}}{2^{3/2}} = \frac{1}{2}$
%%%%%
%%%%%
\end{document}
