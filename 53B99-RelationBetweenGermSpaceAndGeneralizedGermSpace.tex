\documentclass[12pt]{article}
\usepackage{pmmeta}
\pmcanonicalname{RelationBetweenGermSpaceAndGeneralizedGermSpace}
\pmcreated{2013-03-22 19:18:23}
\pmmodified{2013-03-22 19:18:23}
\pmowner{joking}{16130}
\pmmodifier{joking}{16130}
\pmtitle{relation between germ space and generalized germ space}
\pmrecord{5}{42243}
\pmprivacy{1}
\pmauthor{joking}{16130}
\pmtype{Theorem}
\pmcomment{trigger rebuild}
\pmclassification{msc}{53B99}

\endmetadata

% this is the default PlanetMath preamble.  as your knowledge
% of TeX increases, you will probably want to edit this, but
% it should be fine as is for beginners.

% almost certainly you want these
\usepackage{amssymb}
\usepackage{amsmath}
\usepackage{amsfonts}

% used for TeXing text within eps files
%\usepackage{psfrag}
% need this for including graphics (\includegraphics)
%\usepackage{graphicx}
% for neatly defining theorems and propositions
%\usepackage{amsthm}
% making logically defined graphics
%%%\usepackage{xypic}

% there are many more packages, add them here as you need them

% define commands here

\begin{document}
Let $X$, $Y$ be a topological spaces and $x\in X$. Consider \PMlinkname{the germ space}{GermSpace} and \PMlinkname{the generalized germ space}{GermSpace} at $x$:
$$G_x(X,Y);\ \ G_x^*(X,Y).$$
If $f:X\to Y$ is a continuous function, then we have an induced element $[f]\in G_x^*(X,Y)$. It can be easily seen, that if $[f]=[g]\in G_x(X,Y)$, then $[f]=[g]\in G_x^*(X,Y)$. In particular we have a well-defined mapping
$$\tau:G_x(X,Y)\to G_x^*(X,Y);$$
$$\tau([f])=[f].$$

\textbf{Proposition 1.} $\tau$ is injective.

\textit{Proof.} Indeed, assume that $\tau([f])=\tau([g])$ for some $f,g:X\to Y$. Let $f':U\to Y$ and $g':U'\to Y$ be a representatives of $\tau([f])$ and $\tau([g])$ respectively. It follows, that there exists an open neighbourhood $V\subseteq X$ of $x$ such that 
$$f_{|V}=f'_{|V}=g'_{|V}=g_{|V}.$$
In particular $[f]=[g]$ in $G_x(X,Y)$, which completes the proof. $\square$

\textbf{Proposition 2.} If $X$ is a normal space and $Y$ is a normal absolute retract (for example $Y=\mathbb{R}$), then $\tau$ is onto.

\textit{Proof.} Assume that $[f]\in G_x^*(X,Y)$ for some $f:U\to Y$. Since $X$ is regular (because it is normal) then there exists an open neighbourhood $V\subseteq X$ such that the closure $\overline{V}\subseteq U$. Now since $X$ is normal and $Y$ is a normal absolute retract, then $f_{|\overline{V}}$ can be extended to entire $X$ (by the generalized Tietze extension theorem). It is easily seen that any such extension gives the same element in $G_x(X,Y)$ (and it is independent on the choice of the representative $f$) and if $F:X\to Y$ is an extension of $f_{|\overline{V}}$, then
$$\tau([F])=[f]$$
because $F_{|V}=f_{|V}$. This completes the proof. $\square$

\textbf{Remark.} If in addition $Y$ is a topological ring (for example $Y=\mathbb{R}$), then it can be easily checked that $\tau$ preserves ring structures. In particular if $X$ is normal and $Y=\mathbb{R}$ or $Y=\mathbb{C}$, then $\tau$ is an isomorphism of rings. Also it is a good question whether the assumptions in proposition 2 can be weakened.
%%%%%
%%%%%
\end{document}
