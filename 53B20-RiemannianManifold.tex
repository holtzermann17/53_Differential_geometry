\documentclass[12pt]{article}
\usepackage{pmmeta}
\pmcanonicalname{RiemannianManifold}
\pmcreated{2013-03-22 13:02:54}
\pmmodified{2013-03-22 13:02:54}
\pmowner{djao}{24}
\pmmodifier{djao}{24}
\pmtitle{Riemannian manifold}
\pmrecord{31}{33452}
\pmprivacy{1}
\pmauthor{djao}{24}
\pmtype{Definition}
\pmcomment{trigger rebuild}
\pmclassification{msc}{53B20}
\pmclassification{msc}{53B21}
\pmsynonym{Riemann space and metric}{RiemannianManifold}
%\pmkeywords{Riemannian metric tensor}
%\pmkeywords{Riemannian manifold}
%\pmkeywords{Riemann space}
%\pmkeywords{Quantum Riemannian Geometry}
%\pmkeywords{category of Riemannian manifolds}
\pmrelated{QuantumGeometry2}
\pmrelated{Gradient}
\pmrelated{CategoryOfRiemannianManifolds}
\pmrelated{HomotopyCategory}
\pmrelated{CWComplexDefinitionRelatedToSpinNetworksAndSpinFoams}
\pmrelated{QuantumAlgebraicTopologyOfCWComplexRepresentationsNewQATResultsForQuantumStateSpacesOfSpinNetworks}
\pmdefines{Riemannian metric}
\pmdefines{Riemannian structure}
\pmdefines{metric tensor}

% this is the default PlanetMath preamble.  as your knowledge
% of TeX increases, you will probably want to edit this, but
% it should be fine as is for beginners.

% almost certainly you want these
\usepackage{amssymb}
\usepackage{amsmath}
\usepackage{amsfonts}
\newcommand{\rT}{\operatorname{T}}
\newcommand{\Rset}{\mathbb{R}}
\newcommand{\ddx}[1]{\frac{\partial}{\partial  x^{#1}}}
\newcommand{\iprod}{\mathop{\rfloor}}


% used for TeXing text within eps files
%\usepackage{psfrag}
% need this for including graphics (\includegraphics)
%\usepackage{graphicx}
% for neatly defining theorems and propositions
%\usepackage{amsthm}
% making logically defined graphics
%%%\usepackage{xypic} 

% there are many more packages, add them here as you need them

% define commands here
\begin{document}
A {\em Riemannian metric tensor} is a covariant, type $(0,2)$ tensor
field 
$g\in\Gamma(\rT^* M\otimes \rT^*M)$ such that at each point $p\in M$,
the bilinear form
$g_p:\rT_pM\times \rT_p M\to \Rset$ is symmetric and
positive definite. Here $T^* M$ is the cotangent bundle of $M$ (defined as a sheaf), $\Gamma$ is the set of global sections of $T^* M \otimes T^* M$, and $g_p$ is the value of the function $g$ at the point $p \in M$.

Let $(x^1,\ldots,x^n)$ be a system of local coordinates on an open
subset $U\subset M$, let $dx^i,\; i=1,\ldots, n$ be the corresponding
coframe of 1-forms, and let $\displaystyle \ddx{i},\; i=1,\ldots, n$
be the corresponding dual frame of vector fields.  Using the local
coordinates, the metric tensor has the unique expression
$$g=\sum_{i,j=1}^n g_{ij}\, dx^i\otimes dx^j,$$
where the metric
tensor components $$g_{ij}=g\left(\ddx{i},\ddx{j}\right)$$
are smooth
functions on $U$.

Once we fix the local coordinates, the functions $g_{ij}$ completely
determine the Riemannian metric.  Thus, at each point $p\in U$, the
matrix $(g_{ij}(p))$ is symmetric, and positive definite.  Indeed, it
is possible to define a Riemannian structure on a manifold $M$ by
specifying an atlas over $M$ together with a matrix of functions
$g_{ij}$ on each coordinate chart which are symmetric and positive
definite, with the proviso that the $g_{ij}$'s must be compatible with
each other on overlaps.

A manifold $M$ together with a Riemannian metric tensor $g$ is called
a {\em Riemannian manifold}.

{\bf Note:} A Riemannian metric tensor on $M$ is not a distance metric
on $M$. However, on a connected manifold every Riemannian metric
tensor on $M$ induces a distance metric on $M$, given by $$
d(x,y) :=
\inf \left\{ \int_0^1 \left[g\!\!\left( \frac{dc}{dt},
      \frac{dc}{dt}\right)_{\!c(t)}\right]^{1/2}dt \right\} ,\quad
x,y\in M,$$
where the infimum is taken over all rectifiable curves
$c:[0,1]\to M$ with $c(0)=x$ and $c(1)=y$.

Often, it is the $g_{ij}$ that are referred to as the ``Riemannian
metric''.  This, however, is a misnomer.  Properly speaking, the
$g_{ij}$ should be called local coordinate components of a metric
tensor, where as ``Riemannian metric'' should refer to the distance
function defined above.  However, the practice of calling the
collection of $g_{ij}$'s by the misnomer ``Riemannian metric'' appears
to have stuck.

\textbf{Remarks:}
\begin{itemize}

\item Both the Riemannian manifold and Riemannian metric tensor are fundamental concepts
in Einstein's General Relativity (GR) theory where the ``Riemannian metric'' and curvature of the
physical Riemannian space-time are changed by the presence of massive bodies and energy
according to \PMlinkname{Einstein's fundamental GR field equations}{EinsteinFieldEquations}. 
\item The category of Riemannian manifolds (or `spaces') provides an alternative framework for GR theories
as well as algebraic quantum field theories (AQFTs);
\item The category of `pseudo-Riemannian' manifolds, deals in fact with extensions of Minkowski spaces, does not
possess the Riemannian metric defined in this entry on Riemannian manifolds, and is claimed as a useful
approach to defining 4D-spacetimes in relativity theories.

\end{itemize}
%%%%%
%%%%%
\end{document}
