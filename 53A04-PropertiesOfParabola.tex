\documentclass[12pt]{article}
\usepackage{pmmeta}
\pmcanonicalname{PropertiesOfParabola}
\pmcreated{2013-03-22 18:54:05}
\pmmodified{2013-03-22 18:54:05}
\pmowner{pahio}{2872}
\pmmodifier{pahio}{2872}
\pmtitle{properties of parabola}
\pmrecord{17}{41749}
\pmprivacy{1}
\pmauthor{pahio}{2872}
\pmtype{Topic}
\pmcomment{trigger rebuild}
\pmclassification{msc}{53A04}
\pmclassification{msc}{51N20}
\pmrelated{ThreeTheoremsOnParabolas}
\pmrelated{PropertiesOfEllipse}
\pmrelated{Hyperbola2}

% this is the default PlanetMath preamble.  as your knowledge
% of TeX increases, you will probably want to edit this, but
% it should be fine as is for beginners.

% almost certainly you want these
\usepackage{amssymb}
\usepackage{amsmath}
\usepackage{amsfonts}
\usepackage{amsthm}

\usepackage{mathrsfs}
\usepackage{pstricks}
\usepackage{pst-plot}

% used for TeXing text within eps files
%\usepackage{psfrag}
% need this for including graphics (\includegraphics)
%\usepackage{graphicx}
% for neatly defining theorems and propositions
%
% making logically defined graphics
%%%\usepackage{xypic}

% there are many more packages, add them here as you need them

% define commands here

\newcommand{\sR}[0]{\mathbb{R}}
\newcommand{\sC}[0]{\mathbb{C}}
\newcommand{\sN}[0]{\mathbb{N}}
\newcommand{\sZ}[0]{\mathbb{Z}}

 \usepackage{bbm}
 \newcommand{\Z}{\mathbbmss{Z}}
 \newcommand{\C}{\mathbbmss{C}}
 \newcommand{\F}{\mathbbmss{F}}
 \newcommand{\R}{\mathbbmss{R}}
 \newcommand{\Q}{\mathbbmss{Q}}



\newcommand*{\norm}[1]{\lVert #1 \rVert}
\newcommand*{\abs}[1]{| #1 |}



\newtheorem{thm}{Theorem}
\newtheorem{defn}{Definition}
\newtheorem{prop}{Proposition}
\newtheorem{lemma}{Lemma}
\newtheorem{cor}{Corollary}
\begin{document}
Let us consider the parabola
\begin{align}
y \;=\; \frac{x^2}{2p}
\end{align}
where $2p$ is the latus rectum (a.k.a. parametre), i.e. the double distance of the focus from the directrix.\\

\textbf{1.}\, Cut the parabola with the family
\begin{align}
y \;=\; mx\!+\!k
\end{align}
of parallel lines ($m$ is \PMlinkname{constant}{Polynomial}).\, Substituting the right hand side of (2) into (1) yields the quadratic equation
$$x^2-2mpx-2kp \;=\; 0$$
which determines the abscissas of the intersection points.\, By the properties of quadratic equations, the sum of abscissas of both points is $2mp$ and thus their arithmetic mean is $mp$.\, This means that the midpoint of the chord cut by the parabola from the line (2) has the constant abscissa
\begin{align}
x_0 \;=\; mp.
\end{align}
Accordingly, \emph{all midpoints of parallel chords of parabola are on a line parallel to the axis of parabola}.\\

\textbf{2.}\, In the case the line (2) is a tangent of the parabola, the midpoint $P_0$ of the chord coincides with the point of tangency, having the abscissa $x_0$.\, Thus the slope of the tangent is by (3) equal
$$m_t \;=\; \frac{x_0}{p}$$
and therefore the equation of the tangent is
$$y\!-\!y_0 \;=\; \frac{x_0}{p}(x\!-\!x_0) \;\equiv\; \frac{x_0x}{p}-2\!\cdot\!\frac{x_0^2}{2p} 
\;\equiv\; \frac{x_0x}{p}-2y_0.$$
This is simplified to
\begin{align}
y\!+\!y_0 \;=\; \frac{x_0x}{p}
\end{align}
(cf. tangent of conic section).\\


\textbf{3.}\, The tangent (4) cuts the axis of the parabola in the point $T$ whose ordinate is $-y_0\,(\leqq\, 0)$.\, If 
$F$ is the focus and $N$ the projection of $P_0$ on the directrix, we have
$$TF \;=\; y_0+\frac{p}{2} \;=\; P_0N \;=\; P_0F$$
(the last equality by the definition of parabola).\, Thus we see that the quadrilateral $P_0FTN$ is a rhombus.\, Therefore its diagonals bisect and intersect each other perpendicularly (in the point $M$ on the $x$-axis).\, Consequently, we get the following two results.\\

\emph{The projection of the focus on any tangent of parabola is on the tangent whose point of tangency is the apex of the parabola.}\\

\emph{The tangent of parabola forms equal angles with the axis and the focal radius drawn to the point of tangency.}


\begin{center}
\begin{pspicture}(-5,-3.5)(5,7)
\psaxes[Dx=10,Dy=10]{->}(0,0)(-4.5,-3)(4.5,5)
\rput(4.6,-0.2){$x$}
\rput(0.2,5.1){$y$}
\psplot[linecolor=blue]{-4}{4}{x x mul 4 div}
\psline (-4,-1)(4,-1)
\psline[linecolor=blue](4,3.75)(-0.4,-2.85)
\psline(0,1)(3,2.25)
\psline(3,2.25)(3,-1)(0,-2.25)
\psline(0,1)(3,-1)
\psline[linestyle=dotted](0,2.25)(3,2.25)
\psdots(0,1)(3,2.25)(3,-1)(0,-2.25)(1.5,0)
\rput(0.23,1.3){$F$}
\rput(-0.2,2.25){$y_0$}
\rput(-0.4,-2.2){$-y_0$}
\rput(0.23,-2.4){$T$}
\rput(3.3,2.25){$P_0$}
\rput(3.2,-1.2){$N$}
\rput(1.8,0.2){$M$}
\rput(-2.8,-1.3){directrix}
\rput(-3,4){$y\,=\,\frac{x^2}{2p}$}
\end{pspicture}
\end{center}

\begin{thebibliography}{8}
\bibitem{LP}{\sc Lauri Pimi\"a}: {\em Analyyttinen geometria}.\, Werner S\"oderstr\"om Osakeyhti\"o, Porvoo and Helsinki (1958).
\end{thebibliography} 


%%%%%
%%%%%
\end{document}
