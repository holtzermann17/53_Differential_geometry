\documentclass[12pt]{article}
\usepackage{pmmeta}
\pmcanonicalname{Catenary}
\pmcreated{2014-10-26 21:25:30}
\pmmodified{2014-10-26 21:25:30}
\pmowner{pahio}{2872}
\pmmodifier{pahio}{2872}
\pmtitle{catenary}
\pmrecord{29}{37146}
\pmprivacy{1}
\pmauthor{pahio}{2872}
\pmtype{Derivation}
\pmcomment{trigger rebuild}
\pmclassification{msc}{53B25}
\pmclassification{msc}{51N05}
\pmsynonym{chain curve}{Catenary}
\pmrelated{EquationOfCatenaryViaCalculusOfVariations}
\pmrelated{LeastSurfaceOfRevolution}
\pmrelated{HyperbolicFunctions}
\pmrelated{Tractrix}
\pmrelated{EqualArcLengthAndArea}
\pmdefines{catenary}

\endmetadata

% this is the default PlanetMath preamble.  as your knowledge
% of TeX increases, you will probably want to edit this, but
% it should be fine as is for beginners.

% almost certainly you want these
\usepackage{amssymb}
\usepackage{amsmath}
\usepackage{amsfonts}

% used for TeXing text within eps files
%\usepackage{psfrag}
% need this for including graphics (\includegraphics)
\usepackage{graphicx}
% for neatly defining theorems and propositions
 \usepackage{amsthm}
% making logically defined graphics
%%%\usepackage{xypic}

% there are many more packages, add them here as you need them

% define commands here

\theoremstyle{definition}
\newtheorem*{thmplain}{Theorem}
\begin{document}
A \PMlinkescapetext{chain or a homogeneous flexible thin} wire takes a form resembling an arc of a parabola when suspended at its ends.\, The arc is not from a parabola but from the graph of the \PMlinkname{hyperbolic cosine}{HyperbolicFunctions} function in a suitable coordinate system.

Let's derive the equation \,$y = y(x)$\, of this curve, called the {\em catenary}, in its plane with $x$-axis horizontal and $y$-axis vertical.\, We denote the \PMlinkescapetext{line density of the weight} of the wire by $\sigma$.

In any point \,$(x,\,y)$\, of the wire, the tangent line of the curve forms an angle $\varphi$ with the positive direction of $x$-axis.\, Then,
             $$\tan\varphi \;=\; \frac{dy}{dx} \;=\; y'.$$
In the point, a certain tension $T$ of the wire acts in the direction of the \PMlinkescapetext{tangent; it has the horizontal component\, $T\cos\varphi$\, which has apparently a constant} value $a$.\, Hence we may write
              $$T \;=\; \frac{a}{\cos\varphi},$$
whence the vertical \PMlinkescapetext{component} of $T$ is
          $$T\sin{\varphi} \;=\; a\tan{\varphi}$$
and its \PMlinkname{differential}{Differential}
      $$d(T\sin{\varphi}) \;=\; a\,d\tan{\varphi} \;=\; a\,dy'.$$
But this differential is the amount of the supporting \PMlinkescapetext{force acting on an infinitesimal portion of the wire having the projection $dx$ on the $x$-axis.\, Because of the equilibrium, this force must be equal the weight}\, $\sigma\sqrt{1+(y'(x))^2}\,dx$ (see the arc length).\, Thus we obtain the differential equation
\begin{align}
\sigma\sqrt{1\!+\!y'^2}\,dx \;=\; a\,dy',
\end{align}
which allows the separation of variables:
$$\int dx \;=\; \frac{a}{\sigma}\int\frac{dy'}{\sqrt{1\!+\!y'^2}}$$
This may be solved by using the \PMlinkname{substitution}{SubstitutionForIntegration}
$$y' \;:=\; \sinh{t}, \qquad dy' \;=\; \cosh{t}\,dt, \qquad \sqrt{1\!+\!y'^2} \;=\; \cosh{t}$$
giving
$$x \;=\; \frac{a}{\sigma}t+x_0,$$
i.e.
$$y' \;=\; \frac{dy}{dx} \;=\; \sinh\frac{\sigma(x\!-\!x_0)}{a}.$$
This leads to the final solution
$$y \;=\; \frac{a}{\sigma}\cosh\frac{\sigma(x\!-\!x_0)}{a}+y_0$$
of the equation (1).\, We have denoted the constants of integration by $x_0$ and $y_0$.\, They determine the position of the catenary in regard to the coordinate axes.\, By a suitable choice of the axes and the \PMlinkescapetext{measure units one gets the simple} equation
\begin{align}
y \;=\; a\cosh\frac{x}{a}
\end{align}
of the catenary.

\begin{center}
\includegraphics{catenary}
\end{center}

\textbf{Some \PMlinkescapetext{properties} of catenary}
\begin{itemize}
\item $\tan\varphi = \sinh\frac{x}{a}, \quad \sin\varphi = \tanh\frac{x}{a}$ \quad (cf. the Gudermannian)
\item The arc length of the catenary (2) from the apex\, $(0,\,a)$\, to the point\, $(x,\,y)$\, is\,\, $a\sinh\frac{x}{a} = \sqrt{y^2\!-\!a^2}$.
\item The radius of curvature of the catenary (2) is\, $a\cosh^2\frac{x}{a}$, which is the same as length of the normal line of the catenary between the curve and the $x$-axis.
\item The catenary is the catacaustic of the \PMlinkname{exponential curve}{ExponentialFunction} reflecting the vertical rays.
\item If a parabola rolls on a straight line, the focus draws a catenary.
\item The involute (a.k.a. the evolvent) of the catenary is the tractrix.
\end{itemize}
%%%%%
%%%%%
\end{document}
