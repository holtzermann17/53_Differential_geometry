\documentclass[12pt]{article}
\usepackage{pmmeta}
\pmcanonicalname{Curl}
\pmcreated{2013-03-22 12:47:39}
\pmmodified{2013-03-22 12:47:39}
\pmowner{rspuzio}{6075}
\pmmodifier{rspuzio}{6075}
\pmtitle{curl}
\pmrecord{17}{33110}
\pmprivacy{1}
\pmauthor{rspuzio}{6075}
\pmtype{Definition}
\pmcomment{trigger rebuild}
\pmclassification{msc}{53-01}
\pmsynonym{rotor}{Curl}
%\pmkeywords{vector analysis}
%\pmkeywords{Stokes' theorem}
\pmrelated{IrrotationalField}
\pmrelated{FirstOrderOperatorsInRiemannianGeometry}
\pmrelated{AlternateCharacterizationOfCurl}
\pmrelated{ExampleOfLaminarField}
\pmdefines{curl of a vector field}

% this is the default PlanetMath preamble.  as your knowledge
% of TeX increases, you will probably want to edit this, but
% it should be fine as is for beginners.

% almost certainly you want these
\usepackage{amssymb}
\usepackage{amsmath}
\usepackage{amsfonts}

% used for TeXing text within eps files
%\usepackage{psfrag}
% need this for including graphics (\includegraphics)
%\usepackage{graphicx}
% for neatly defining theorems and propositions
%\usepackage{amsthm}
% making logically defined graphics
%%%\usepackage{xypic}

% there are many more packages, add them here as you need them

% define commands here
\begin{document}
The \emph{curl} (also known as \emph{rotor}) is a first order linear
differential operator which acts on vector fields in $\mathbb{R}^{3}$.

Intuitively, the curl of a vector field measures the extent to which a
vector field differs from being the gradient of a scalar field.  The
name "curl" comes from the fact that vector fields at a point with a
non-zero curl can be seen as somehow "swirling around" said point.  A
mathematically precise formulation of this notion can be obtained in
the form of the definition of curl as limit of an integral about a
closed circuit.

Let $F$ be a  vector field in $\mathbb{R}^{3}$.

Pick an orthonormal basis $\{\vec{e_{1}},\vec{e_{2}},\vec{e_{3}}\}$
and write
$\vec{F}=F^{1}\vec{e_{1}}+F^{2}\vec{e_{2}}+F^{3}\vec{e_{3}}$.  Then
the curl of $F$, notated $\operatorname{curl}\vec{F}$ or
$\operatorname{rot}\vec{F}$ or $\vec{\nabla}\times\vec{F}$, is given
as follows:
 
\begin{eqnarray*}
\operatorname{curl}\vec{F} & = & \left[\frac{\partial F^{3}}{\partial
q^{2}}-\frac{\partial F^{2}}{\partial
q^{3}}\right]\vec{e_{1}}+\left[\frac{\partial F^{1}}{\partial
q^{3}}-\frac{\partial F^{3}}{\partial
q^{1}}\right]\vec{e_{2}} + \\ 
 & \; & \left[\frac{\partial F^{2}}{\partial 
q^{1}}-\frac{\partial F^{1}}{\partial 
q^{2}}\right]\vec{e_{3}}
\end{eqnarray*}

By applying the chain rule, one can verify that one obtains the same
answer irregardless of choice of basis, hence curl is well-defined as
a function of vector fields.  Another way of coming to the same
conclusion is to exhibit an expression for the curl of a vector field
which does not require the choice of a basis.  One such expression is
as follows: Let $V$ be the volume of a closed surface $S$ enclosing
the point $p$.  Then one has

\[
\operatorname{curl}\vec{F}(p)=\lim_{V\to
0}\frac{1}{V}\int\!\!\int_{S}\vec{n}\times\vec{F}dS
\]

Where $n$ is the outward unit normal vector to $S$.

Curl is easily computed in an
arbitrary orthogonal coordinate system by using the appropriate
scale factors. That is


\begin{eqnarray*}
\operatorname{curl}\vec{F} & = & \frac{1}{h_{3}h_{2}}\left[\frac{\partial}{\partial
q^{2}}\left(h_{3}F^{3}\right)-\frac{\partial}{\partial
q^{3}}\left(h_{2}F^{2}\right)\right]\vec{e_{1}}+\frac{1}{h_{3}h_{1}}\left[\frac{\partial}{\partial
q^{3}}\left(h_{1}F^{1}\right)-\frac{\partial}{\partial
q^{1}}\left(h_{3}F^{3}\right)\right]\vec{e_{2}} + \\ 
 & \; & \frac{1}{h_{1}h_{2}}\left[\frac{\partial}{\partial
q^{1}}\left(h_{2}F^{2}\right)-\frac{\partial}{\partial
q^{2}}\left(h_{1}F^{1}\right)\right]\vec{e_{3}}
\end{eqnarray*}

for the arbitrary orthogonal curvilinear coordinate system
$(q^{1},q^{2},q^{3})$ having scale factors $(h_{1},h_{2},h_{3})$.
Note the scale factors are given by

\[
h_{i}=\left(\frac{d}{dx_{i}}\right)\left(\frac{d}{dx_{i}}\right)\;\ni\;
i\in \{1,2,3\}.
\]

 Non-orthogonal systems are more easily handled with
tensor analysis or exterior calculus. 

\[ (\operatorname{curl}\vec{F})^i = \epsilon^{ijk} \nabla_j F_k \]

\[ \operatorname{curl}\vec{F} = * d (F_1 dx^1 + F_2 dx^2 + F_3 dx^3) \]
%%%%%
%%%%%
\end{document}
