\documentclass[12pt]{article}
\usepackage{pmmeta}
\pmcanonicalname{Curve}
\pmcreated{2013-03-22 12:54:17}
\pmmodified{2013-03-22 12:54:17}
\pmowner{rmilson}{146}
\pmmodifier{rmilson}{146}
\pmtitle{curve}
\pmrecord{28}{33255}
\pmprivacy{1}
\pmauthor{rmilson}{146}
\pmtype{Definition}
\pmcomment{trigger rebuild}
\pmclassification{msc}{53B25}
\pmclassification{msc}{14H50}
\pmclassification{msc}{14F35}
\pmclassification{msc}{51N05}
\pmsynonym{parametrized curve}{Curve}
\pmsynonym{parameterized curve}{Curve}
\pmsynonym{path}{Curve}
\pmsynonym{trajectory}{Curve}
\pmrelated{FundamentalGroup}
\pmrelated{TangentSpace}
\pmrelated{RealTree}
\pmdefines{closed curve}
\pmdefines{Jordan curve}
\pmdefines{regular curve}
\pmdefines{simple closed curve}
\pmdefines{simple curve}
\pmdefines{plane curve}
\pmdefines{planar curve}
\pmdefines{convex curve}
\pmdefines{locally convex curve}
\pmdefines{local multiplicity}
\pmdefines{globally  convex}
\pmdefines{global multiplicity}

\endmetadata

\usepackage{amsmath}
\usepackage{amsfonts}
\usepackage{amssymb}

\newcommand{\Rset}{\mathbb{R}}
\newcommand{\lp}{\left(}
\newcommand{\rp}{\right)}
\newcommand{\lb}{\left[}
\newcommand{\rb}{\right]}
\begin{document}
\PMlinkescapeword{regular}
\PMlinkescapeword{simple}
\PMlinkescapeword{term}
\section*{Summary.}
The term \emph{curve} is associated with two closely related notions.
The first notion is kinematic: a parameterized curve is a function of
one real variable taking values in some ambient geometric setting.
This variable can be interpreted as time, in which case the function
describes the evolution of a moving particle. The second notion is
geometric; in this sense a curve is an arc, a 1-dimensional subset of
an ambient space. The two notions are related: the image of a
parameterized curve describes the trajectory of a moving particle.
Conversely, a given arc admits multiple parameterizations.  A
trajectory can be traversed by moving particles at different speeds.

In algebraic geometry, the term curve is used to describe a
1-dimensional variety relative to the complex numbers or some other
ground field.  This can be potentially confusing, because a curve over
the complex numbers refers to an object which, in conventional
geometry, one would refer to as a surface. In particular, a Riemann
surface can be regarded as as complex curve.

\section*{Kinematic definition}
Let $I\subset \Rset$ be an \PMlinkname{interval}{Interval} of the real line. A parameterized
curve is a continuous mapping $\gamma:I\to X$ taking values in a
topological space $X$. We say that $\gamma$ is a \emph{simple curve}
if it has no self-intersections, that is if the mapping $\gamma$ is
injective.

We say that $\gamma$ is a \emph{closed curve}, or a
\emph{\PMlinkname{loop}{loop}} whenever $I=[a,b]$ is a closed
interval, and the endpoints are mapped to the same value;
$\gamma(a)=\gamma(b).$ Equivalently, a loop may be defined to be a
continuous mapping $\gamma \colon \mathbb{S}^1\to X$ whose domain
$\mathbb{S}^1$ is the unit circle. A simple closed curve is often
called a \emph{Jordan curve}.

If $X=\mathbb{R}^2$ then $\gamma$ is called a \emph{plane curve} or \emph{planar curve}.

A smooth closed curve $\gamma$ in $\mathbb{R}^n$ is \emph{locally \PMlinkescapetext{convex}} 
if the local multiplicity of intersection  
of $\gamma$ with each hyperplane at of each of the intersection points does not
exceed $n$. The \emph{global multiplicity} is the sum of the local
multiplicities. 
A simple smooth curve in $\mathbb{R}^n$ is called \emph{\PMlinkescapetext{convex}} (or 
\emph{globally \PMlinkescapetext{convex}}) if the global multiplicity
of its intersection with any affine hyperplane is less than or equal to $n$.
An example of a closed convex curve in $\mathbb{R}^{2n}$ is the normalized
generalized ellipse:
$$
(\sin t, \cos t, \frac{\sin 2t}{2}, \frac{\cos 2t}{2}, \ldots , \frac{\sin nt}{n}, \frac{\cos nt}{n}).
$$
In odd dimension there are no closed convex curves. 


In many instances the ambient space $X$ is a differential manifold, in
which case we can speak of differentiable curves. Let $I$ be an open
interval, and let $\gamma:I\to X$ be a differentiable curve. For
every $t\in I$ can regard the \PMlinkname{derivative}{RelatedRates},
$\dot{\gamma}(t)$, as the \PMlinkname{velocity}{RelatedRates} of a
moving particle, at time $t$. The velocity $\dot{\gamma}(t)$ is a
\PMlinkname{tangent vector}{TangentSpace}, which belongs to
$T_{\gamma(t)} X$, the tangent space of the manifold $X$ at the point
$\gamma(t)$. We say that a differentiable curve $\gamma(t)$ is
\emph{regular}, if its velocity, $\dot{\gamma}(t)$, is non-vanishing
for all $t\in I$.

It is also quite common to consider curves that take values in
$\Rset^n$. In this case, a parameterized curve can be regarded as a
vector-valued function $\vec{\gamma}:I \to \Rset^n$, that is an
  $n$-tuple of functions $$\vec{\gamma}(t) =
\begin{pmatrix}
\gamma_1(t)\\ \vdots \\ \gamma_n(t)
\end{pmatrix},$$
where $\gamma_i:I\to \Rset$, $i=1,\ldots,n$ are scalar-valued functions.

\section*{Geometric definition.}
A (non-singular) curve $C$, equivalently, an arc, is a connected,
1-dimensional submanifold of a differential manifold $X$. This means
that for every point $p\in C$ there exists an open neighbourhood
$U\subset X$ of $p$ and a chart $\alpha:U\to \Rset^n$ such that
$$\alpha(C\cap U) = \{ (t,0,\ldots,0)\in \Rset^n :
-\epsilon<t<\epsilon\}$$
for some real $\epsilon>0$.

An alternative, but equivalent definition, describes an arc as the
image of a regular parameterized curve. To accomplish this, we need
to define the notion of reparameterization. Let $I_1,I_2\subset
\Rset$ be intervals. A reparameterization is a continuously
differentiable function
$$s:I_1\to I_2$$ whose derivative is never vanishing. Thus, $s$ is
either monotone increasing, or monotone decreasing.
Two regular, parameterized curves
$$\gamma_i:I_i\to X,\quad i=1,2$$
are said to be related by a reparameterization if there exists a
reparameterization $s:I_1\to I_2$ such that
$$\gamma_1 = \gamma_2\circ s.$$
The inverse of a reparameterization
function is also a reparameterization. Likewise, the composition of
two parameterizations is again a reparameterization. Thus the
reparameterization relation between curves, is in fact an
equivalence relation. An arc can now be defined as an equivalence
class of regular, simple curves related by reparameterizations. In
order to exclude pathological embeddings with wild endpoints we also
impose the condition that the arc, as a subset of $X$, be
homeomorphic to an open interval.
%%%%%
%%%%%
\end{document}
